\documentclass[a4paper,11pt]{article}
\usepackage[utf8x]{inputenc}
\usepackage{fancyhdr}
\usepackage[spanish]{babel}
\usepackage{lastpage}
\usepackage{amstext}
\usepackage{amsmath}
\usepackage{amsfonts}
\usepackage{amsthm}
\usepackage{amssymb}
\usepackage{enumerate}
\usepackage{graphicx}
\usepackage{etoolbox}
\usepackage[implicit=false]{hyperref}
\usepackage[a4paper, total={6.5in, 9.5in}]{geometry}
\usepackage[T1]{fontenc}
\usepackage[sc]{mathpazo}

\newcommand{\at}{@}

\title{Movimiento Browniano\\
      \small{Ejercicios entregables - Lista 3}}
\author{Lucio Santi\\
        \texttt{lsanti\at dc.uba.ar}}
\date{\today}

\pagestyle{fancyplain} 
\renewcommand{\headrulewidth}{0pt}
\cfoot{\thepage/\pageref{LastPage}}
\lhead{}
\chead{}
\rhead{}

\newcommand{\abs}[1]{\ensuremath{\left\lvert #1 \right\rvert}}
\newcommand{\Sig}[1]{\ensuremath{\mathcal{#1}}}
\newcommand{\SigAlg}[1]{\ensuremath{\sigma\left(#1\right)}}
\newcommand{\Mart}[2]{\ensuremath{\left(#1_n, \Sig{#2}_n\right)}}
\newcommand{\Exp}[1]{\ensuremath{\textrm{E}\left[#1\right]}}
\newcommand{\ExpC}[2]{\ensuremath{\textrm{E}\left[#1 \, | \, #2\right]}}
\newcommand{\Prob}[1]{\ensuremath{\mathbb{P} \left\{ #1 \right\}}}
\newcommand{\Probx}[2]{\ensuremath{\mathbb{P}_{#1} \left\{ #2 \right\}}}

\newtheorem*{ej}{Ejercicio}

\begin{document}
\maketitle

\begin{ej} 
(2.6 - Mörters y Peres) Sea $(B(t),~ 0 \leq t \leq 1)$ un movimiento browniano lineal y
$$\tau = \sup \left\{ t \in [0,1] : B(t) = 0 \right\}$$
Probar que, casi seguramente, existen tiempos $t_n < s_n < \tau$
con $t_n \uparrow \tau$ tales que $B(t_n) < 0$ y $B(s_n) > 0$.
\end{ej}

\begin{proof}[Resoluci\'on]
Consideremos el proceso $(\tilde{B}(t),~ 0 \leq t \leq \tau)$ en donde
$\tilde{B}(t) = B(\tau - t)$. Es evidente que $\tilde{B}$ es un movimiento browniano
como consecuencia de que $B$ lo sea. Además, $\tilde{B}$ es un movimiento browniano
standard puesto que $\tilde{B}(0) = B(\tau) = 0$ (esto último vale por definición
de $\tau$ y continuidad de $B$). Luego, valiéndonos del resultado estudiado en clase
(que de hecho es enunciado en el Teorema 2.8 del libro), tenemos que
$\Probx{0}{\sigma = 0} = 1$, siendo
$$\sigma = \inf \left\{ 0 < t \leq \tau : \tilde{B}(t) < 0 \right\}$$
De esto sigue que, casi seguramente, podemos encontrar una sucesión
de tiempos $t'_n > 0$ tales que $t'_n \downarrow 0$ y $\tilde{B}(t'_n) < 0$ para todo
$n$. Ahora bien, utilizando el mismo resultado, tenemos que $\Probx{0}{\phi_n = 0} = 1$, 
con
$$\phi_n = \inf \left\{ 0 < t < t'_n : \tilde{B}(t) > 0 \right\}$$
Por ende, podemos afirmar que, casi seguramente, existe una sucesión de tiempos
$r^n_k > 0$ tales que $r^n_k < t'_n$, $r^n_k \downarrow 0$ y $\tilde{B}(r^n_k) > 0$ para todo
$k$. Sea $s'_n = r^n_0$ y sean $t_n = \tau - t'_n$ y $s_n = \tau - s'_n$. De esta forma,
\begin{itemize}
    \item $t_n < s_n$ puesto que $t'_n > s'_n$.
    \item $s_n < \tau$ al ser $s'_n > 0$.
    \item $t_n \uparrow \tau$ puesto que $t'_n \downarrow 0$.
    \item $B(t_n) = \tilde{B}(\tau - t_n) = \tilde{B}(t'_n) < 0$.
    \item $B(s_n) = \tilde{B}(\tau - s_n) = \tilde{B}(s'_n) > 0$.
\end{itemize}
Consecuentemente, las sucesiones $t_n, s_n$ propuestas satisfacen lo solicitado en el enunciado.

\end{proof}

%%%%

\begin{ej}
(2.8 - Mörters y Peres) Probar que, para cualquier $x > 0$ y $A \subset [0,\infty)$ medible,
$$\Probx{x}{B(s) \geq 0 ~\forall~ 0 \leq s \leq t, B(t) \in A} = \Probx{x}{B(t) \in A} - \Probx{-x}{B(t) \in A}$$
\end{ej}

\begin{proof}[Resoluci\'on]
Por la ley de probabilidad total,
$$\Probx{x}{B(t) \in A} = \Probx{x}{B(t) \in A, B(s) \geq 0 ~\forall~ 0 \leq s \leq t} + 
    \Probx{x}{B(t) \in A, \exists~0 \leq s \leq t : B(s) < 0}$$
Veamos entonces que $\Probx{-x}{B(t) \in A} = \Probx{x}{B(t) \in A, \exists~0 \leq s \leq t : B(s) < 0}$. Supongamos
que tenemos un tal browniano $B$ arrancado en $x > 0$ y tal que $B(s) < 0$ para cierto $0 \leq s \leq t$.
Sea $\sigma = \inf \left\{ 0 \leq u \leq t : B(u) = 0 \right\}$. Siendo $B(s) < 0$ y $B(0) > 0$, tal $\sigma$    
existe, es positivo y es tal que $B(\sigma) = 0$ por continuidad de $B$. Consideremos entonces el proceso estocástico
$(\tilde{B}(t),~ t \geq 0)$ definido como sigue, en donde $(B_0(t),~t \geq 0)$ es un movimiento browniano standard:
$$
\tilde{B}(t) = 
\begin{cases}
B(\sigma -t) & \textrm{ si } 0 \leq t \leq \sigma \\
B_0(t - \sigma) + x & \textrm{ si } t > \sigma
\end{cases}
$$
Veamos que $\tilde{B}$ es un movimiento browniano standard:
\begin{itemize}
    \item $\tilde{B}(0) = B(\sigma) = 0$.
    \item Fuera de $t = \sigma$, $\tilde{B}$ es continua por continuidad de $B$ y $B_0$. Además,
    \begin{eqnarray*}
        \lim_{t \to \sigma^{-}}{\tilde{B}(t)} &=& B(0) \\
            &=& x \\
            &=& B_0(0) + x \\
            &=& \lim_{t \to \sigma^{+}}{\tilde{B}(t)}
    \end{eqnarray*}
    \item Dados $t \geq 0$ y $h > 0$, los incrementos $\tilde{B}(t+h) - \tilde{B}(t)$ tienen distribución
    normal con esperanza 0 y varianza $h$ siempre que $t+h$ y $t$ estén ambos a izquierda o a derecha
    de $\sigma$. Cuando $t+h > \sigma$ y $t \leq \sigma$,
    $$\tilde{B}(t+h) - \tilde{B}(t) =
            \underbrace{            
            \underbrace{B_0((t+h) - \sigma)}_{\sim N(0,~t+h-\sigma)} 
            + \underbrace{x - B(\sigma - t)}_{\sim N(0,~\sigma - t)}}_{\sim N(0,~h)}$$
    \item Finalmente, puede verse que $\tilde{B}$ tiene la propiedad de incrementos independientes siguiendo
    un razonamiento similar al del ítem anterior.
\end{itemize}
La idea detrás de $\tilde{B}$ es la de poder reformular la probabilidad anterior de forma más sencilla. 
Esencialmente, $\tilde{B}$ se mueve entre $0$ y $\sigma$ invirtiendo el movimiento de $B$ y luego
comportándose como un browniano standard trasladado en $x$. De esta manera, como consecuencia de la
propiedad de Markov de $B$, el evento de $B$ llegando a $A$ en tiempo $t$ equivale al evento de $B_0$
llegando a $A$ en tiempo $t - \sigma$. Como $B_0$ está trasladado en $x$ y siendo la primera porción del
recorrido de $\tilde{B}$ una inversión del recorrido de $B$, tenemos que
$$\Probx{x}{B(t) \in A, \exists~0 \leq s \leq t : B(s) < 0} = \Probx{0}{\tilde{B}(t) \in A + x}$$
No obstante, llegar a $A+x$ en tiempo $t$ y arrancando en $0$ equivale a llegar a $A$ en el mismo tiempo
pero arrancando en $-x$, con lo cual
$$\Probx{0}{\tilde{B}(t) \in A + x} = \Probx{-x}{B(t) \in A}$$
Esto prueba lo deseado y completa la resolución del ejercicio.

\end{proof}

\end{document}
