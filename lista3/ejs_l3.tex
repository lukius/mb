\documentclass[a4paper,11pt]{article}
\usepackage[utf8x]{inputenc}
\usepackage{fancyhdr}
\usepackage[spanish]{babel}
\usepackage{lastpage}
\usepackage{amstext}
\usepackage{amsmath}
\usepackage{amsfonts}
\usepackage{amsthm}
\usepackage{amssymb}
\usepackage{enumerate}
\usepackage{graphicx}
\usepackage{etoolbox}
\usepackage[implicit=false]{hyperref}
\usepackage[a4paper, total={6.5in, 9.5in}]{geometry}
\usepackage[T1]{fontenc}
\usepackage[sc]{mathpazo}

\newcommand{\at}{@}

\title{Movimiento Browniano\\
      \small{Ejercicios entregables - Lista 3}}
\author{Lucio Santi\\
        \texttt{lsanti\at dc.uba.ar}}
\date{\today}

\pagestyle{fancyplain} 
\renewcommand{\headrulewidth}{0pt}
\cfoot{\thepage/\pageref{LastPage}}
\lhead{}
\chead{}
\rhead{}

\newcommand{\abs}[1]{\ensuremath{\left\lvert #1 \right\rvert}}
\newcommand{\Sig}[1]{\ensuremath{\mathcal{#1}}}
\newcommand{\SigAlg}[1]{\ensuremath{\sigma\left(#1\right)}}
\newcommand{\Mart}[2]{\ensuremath{\left(#1_n, \Sig{#2}_n\right)}}
\newcommand{\Exp}[1]{\ensuremath{\textrm{E}\left[#1\right]}}
\newcommand{\ExpC}[2]{\ensuremath{\textrm{E}\left[#1 \, | \, #2\right]}}
\newcommand{\Prob}[1]{\ensuremath{\mathbb{P} \left\{ #1 \right\}}}
\newcommand{\Probx}[2]{\ensuremath{\mathbb{P}_{#1} \left\{ #2 \right\}}}

\newtheorem*{ej}{Ejercicio}

\begin{document}
\maketitle

\begin{ej} 
(2.6 - Mörters y Peres) Sea $(B(t),~ 0 \leq t \leq 1)$ un movimiento browniano lineal y
$$\tau = \sup \left\{ t \in [0,1] : B(t) = 0 \right\}$$
Probar que, casi seguramente, existen tiempos $t_n < s_n < \tau$
con $t_n \uparrow \tau$ tales que $B(t_n) < 0$ y $B(s_n) > 0$.
\end{ej}

\begin{proof}[Resoluci\'on]
Consideremos el proceso $(\tilde{B}(t),~ 0 \leq t \leq \tau)$ en donde
$\tilde{B}(t) = B(\tau - t)$. Es evidente que $\tilde{B}$ es un movimiento browniano
como consecuencia de que $B$ lo sea. Además, $\tilde{B}$ es un movimiento browniano
standard puesto que $\tilde{B}(0) = B(\tau) = 0$ (esto último vale por definición
de $\tau$ y continuidad de $B$). Luego, valiéndonos del resultado estudiado en clase
(que de hecho es enunciado en el Teorema 2.8 del libro), tenemos que
$\Probx{0}{\sigma = 0} = 1$, siendo
$$\sigma = \inf \left\{ 0 < t \leq \tau : \tilde{B}(t) < 0 \right\}$$
De esto sigue que, casi seguramente, podemos encontrar una sucesión
de tiempos $t'_n > 0$ tales que $t'_n \downarrow 0$ y $\tilde{B}(t'_n) < 0$ para todo
$n$. Ahora bien, utilizando el mismo resultado, tenemos que $\Probx{0}{\phi_n = 0} = 1$, 
con
$$\phi_n = \inf \left\{ 0 < t < t'_n : \tilde{B}(t) > 0 \right\}$$
Por ende, podemos afirmar que, casi seguramente, existe una sucesión de tiempos
$r^n_k > 0$ tales que $r^n_k < t'_n$, $r^n_k \downarrow 0$ y $\tilde{B}(r^n_k) > 0$ para todo
$k$. Sea $s'_n = r^n_0$ y sean $t_n = \tau - t'_n$ y $s_n = \tau - s'_n$. De esta forma,
\begin{itemize}
    \item $t_n < s_n$ puesto que $t'_n > s'_n$.
    \item $s_n < \tau$ al ser $s'_n > 0$.
    \item $t_n \uparrow \tau$ puesto que $t'_n \downarrow 0$.
    \item $B(t_n) = \tilde{B}(\tau - t_n) = \tilde{B}(t'_n) < 0$.
    \item $B(s_n) = \tilde{B}(\tau - s_n) = \tilde{B}(s'_n) > 0$.
\end{itemize}
Consecuentemente, las sucesiones $t_n, s_n$ propuestas satisfacen lo solicitado en el enunciado.

\end{proof}

%%%%

\begin{ej}
(2.8 - Mörters y Peres) Probar que, para cualquier $x > 0$ y $A \subset [0,\infty)$ medible,
$$\Probx{x}{B(s) \geq 0 ~\forall~ 0 \leq s \leq t, B(t) \in A} = \Probx{x}{B(t) \in A} - \Probx{-x}{B(t) \in A}$$
\end{ej}

\begin{proof}[Resoluci\'on]

\end{proof}

\end{document}
