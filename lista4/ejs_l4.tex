\documentclass[a4paper,11pt]{article}
\usepackage[utf8x]{inputenc}
\usepackage{fancyhdr}
\usepackage[spanish]{babel}
\usepackage{lastpage}
\usepackage{amstext}
\usepackage{amsmath}
\usepackage{amsfonts}
\usepackage{amsthm}
\usepackage{amssymb}
\usepackage{enumerate}
\usepackage{graphicx}
\usepackage{etoolbox}
\usepackage[implicit=false]{hyperref}
\usepackage[a4paper, total={6.5in, 9.5in}]{geometry}
\usepackage[T1]{fontenc}
\usepackage[sc]{mathpazo}

\newcommand{\at}{@}

\title{Movimiento Browniano\\
      \small{Ejercicios entregables - Lista 4}}
\author{Lucio Santi\\
        \texttt{lsanti\at dc.uba.ar}}
\date{\today}

\pagestyle{fancyplain} 
\renewcommand{\headrulewidth}{0pt}
\cfoot{\thepage/\pageref{LastPage}}
\lhead{}
\chead{}
\rhead{}

\newcommand{\abs}[1]{\ensuremath{\left\lvert #1 \right\rvert}}
\newcommand{\Sig}[1]{\ensuremath{\mathcal{#1}}}
\newcommand{\SigAlg}[1]{\ensuremath{\sigma\left(#1\right)}}
\newcommand{\Mart}[2]{\ensuremath{\left(#1_n, \Sig{#2}_n\right)}}
\newcommand{\Exp}[1]{\ensuremath{\textrm{E}\left[#1\right]}}
\newcommand{\ExpC}[2]{\ensuremath{\textrm{E}\left[#1 \, | \, #2\right]}}
\newcommand{\Prob}[1]{\ensuremath{\mathbb{P} \left( #1 \right)}}
\newcommand{\Probx}[2]{\ensuremath{\mathbb{P}_{#1} \left( #2 \right)}}
\newcommand{\Expx}[2]{\ensuremath{\textrm{E}_{#1}\left[#2\right]}}
\newcommand{\ExpxC}[3]{\ensuremath{\textrm{E}_{#1}\left[#2 \, | \, #3\right]}}
\newcommand{\Ev}[1]{\ensuremath{\left\{ #1 \right\}}}

\newtheorem*{ej}{Ejercicio}

\begin{document}
\maketitle

\begin{ej} 
Sea $d = 2$, $0 < r < \abs{x} < R$, $\tau = S_r \wedge S_R \,$ y $\, \phi(x) = \log(\abs{x})$,
en donde
$$S_z = \inf \left\{ t > 0 : \abs{B(t)} = z \right\}$$
\begin{enumerate}[a.]
    \item Probar que $\phi(x) = \Expx{x}{\phi(B(\tau))}$.

    \item Probar que
    $$\Probx{x}{S_r < S_R} = \frac{\phi(R) - \phi(\abs{x})}{\phi(R) - \phi(r)}$$

    \item Probar que $\Prob{S_r < \infty} = 1$. Concluir que el movimiento browniano
    bidimensional es recurrente en el sentido de que, para todo
    $G \subset \mathbb{R}^2$ abierto,
    $$\Probx{x}{B \in G \textrm{ infinitas veces}} = 1$$

    \item Definir el evento involucrado en la probabilidad de arriba.

    \item Probar que, para todo $x \neq 0$, $\Probx{x}{S_0 < \infty} = 0$.

    \item Probar que el resultado anterior también vale para $x = 0$.
\end{enumerate}
\end{ej}

\begin{proof}[Resoluci\'on]
$ $ 
\begin{enumerate}[a.]
    \item TBD
    
    \item Para probar esto, desarrollemos $\Expx{x}{\phi(B(\tau))}$ utilizando la ley
    de la esperanza total:
    \begin{eqnarray}
        \Expx{x}{\phi(B(\tau))}
            &=& \Expx{x}{\phi(B(\tau)), S_r < S_R} + 
                \Expx{x}{\phi(B(\tau)), S_r \geq S_R} \nonumber \\
            &=& \ExpxC{x}{\phi(B(\tau))}{S_r < S_R} \, \Probx{x}{S_r < S_R} +
                \ExpxC{x}{\phi(B(\tau))}{S_r \geq S_R} \, \Probx{x}{S_r \geq S_R} \nonumber \\
            &=& \Expx{x}{\phi(B(S_r))}\, \Probx{x}{S_r < S_R} +
                \Expx{x}{\phi(B(S_R))}\, \Probx{x}{S_r \geq S_R} \nonumber \\
            &=& \phi(B(S_r))\, \Probx{x}{S_r < S_R} +
                \phi(B(S_R))\, \Probx{x}{S_r \geq S_R} \nonumber \\
            &=& \log(\abs{B(S_r)})\, \Probx{x}{S_r < S_R} +
                \log(\abs{B(S_R)})\, \Probx{x}{S_r \geq S_R} \nonumber \\
            &=& \log(r)\, \Probx{x}{S_r < S_R} +
                \log(R)\, \Probx{x}{S_r \geq S_R} \nonumber \\
            &=& \log(r)\, \Probx{x}{S_r < S_R} +
                \log(R)\, (1 - \Probx{x}{S_r < S_R}) \nonumber \\
            &=& \Probx{x}{S_r < S_R} (\log(r) - \log(R)) + \log(R) \nonumber \\
            &=& \Probx{x}{S_r < S_R} (\phi(r) - \phi(R)) + \phi(R) \label{eq:E}
    \end{eqnarray}
    Ahora bien, el ítem anterior nos asegura que $\Expx{x}{\phi(B(\tau))} = (\ref{eq:E}) = \phi(x)$,
    por lo que
    $$\Probx{x}{S_r < S_R} = \frac{\phi(x) - \phi(R)}{\phi(r) - \phi(R)} 
        = \frac{\phi(R) - \phi(x)}{\phi(R) - \phi(r)}$$

    \item Al considerar el evento $A = \Ev{S_r < \infty}$, observamos que
    $$\Ev{\union_{n = 1}^{\infty}{\Ev{S_r < S_n}}} \subseteq A$$
\end{enumerate}
\end{proof}

\end{document}
