\documentclass[a4paper,11pt]{article}
\usepackage[utf8x]{inputenc}
\usepackage{fancyhdr}
\usepackage[spanish]{babel}
\usepackage{lastpage}
\usepackage{amstext}
\usepackage{amsmath}
\usepackage{amsfonts}
\usepackage{amsthm}
\usepackage{amssymb}
\usepackage{enumerate}
\usepackage{graphicx}
\usepackage{etoolbox}
\usepackage[implicit=false]{hyperref}
\usepackage[a4paper, total={6.5in, 9.5in}]{geometry}
\usepackage[T1]{fontenc}
\usepackage[sc]{mathpazo}

\newcommand{\at}{@}

\title{Movimiento Browniano\\
      \small{Ejercicios entregables - Lista 4}}
\author{Lucio Santi\\
        \texttt{lsanti\at dc.uba.ar}}
\date{\today}

\pagestyle{fancyplain} 
\renewcommand{\headrulewidth}{0pt}
\cfoot{\thepage/\pageref{LastPage}}
\lhead{}
\chead{}
\rhead{}

\newcommand{\abs}[1]{\ensuremath{\left\lvert #1 \right\rvert}}
\newcommand{\Sig}[1]{\ensuremath{\mathcal{#1}}}
\newcommand{\SigAlg}[1]{\ensuremath{\sigma\left(#1\right)}}
\newcommand{\Mart}[2]{\ensuremath{\left(#1_n, \Sig{#2}_n\right)}}
\newcommand{\Exp}[1]{\ensuremath{\textrm{E}\left[#1\right]}}
\newcommand{\ExpC}[2]{\ensuremath{\textrm{E}\left[#1 \, | \, #2\right]}}
\newcommand{\Prob}[1]{\ensuremath{\mathbb{P} \left( #1 \right)}}
\newcommand{\Probx}[2]{\ensuremath{\mathbb{P}_{#1} \left( #2 \right)}}
\newcommand{\Expx}[2]{\ensuremath{\textrm{E}_{#1}\left[#2\right]}}
\newcommand{\ExpxC}[3]{\ensuremath{\textrm{E}_{#1}\left[#2 \, | \, #3\right]}}
\newcommand{\Ev}[1]{\ensuremath{\left\{ #1 \right\}}}

\newtheorem*{ej}{Ejercicio}

\begin{document}
\maketitle

\begin{ej} 
Sea $d = 2$, $0 < r < \abs{x} < R$, $\tau = S_r \wedge S_R \,$ y $\, \phi(x) = \log(\abs{x})$,
en donde
$$S_z = \inf \left\{ t > 0 : \abs{B(t)} = z \right\}$$
\begin{enumerate}[a.]
    \item Probar que $\phi(x) = \Expx{x}{\phi(B(\tau))}$.

    \item Probar que
    $$\Probx{x}{S_r < S_R} = \frac{\phi(R) - \phi(\abs{x})}{\phi(R) - \phi(r)}$$

    \item Probar que $\Prob{S_r < \infty} = 1$. Concluir que el movimiento browniano
    bidimensional es recurrente en el sentido de que, para todo
    $G \subset \mathbb{R}^2$ abierto,
    $$\Probx{x}{B \in G \textrm{ infinitas veces}} = 1$$

    \item Definir el evento involucrado en la probabilidad de arriba.

    \item Probar que, para todo $x \neq 0$, $\Probx{x}{S_0 < \infty} = 0$.

    \item Probar que el resultado anterior también vale para $x = 0$.
\end{enumerate}
\end{ej}

\begin{proof}[Resoluci\'on]
$ $ 
\begin{enumerate}[a.]
    \item TBD
    
    \item Para probar esto, desarrollemos $\Expx{x}{\phi(B(\tau))}$ utilizando la ley
    de la esperanza total:
    \begin{eqnarray}
        \Expx{x}{\phi(B(\tau))}
            &=& \Expx{x}{\phi(B(\tau)), S_r < S_R} + 
                \Expx{x}{\phi(B(\tau)), S_r \geq S_R} \nonumber \\
            &=& \ExpxC{x}{\phi(B(\tau))}{S_r < S_R} \, \Probx{x}{S_r < S_R} +
                \ExpxC{x}{\phi(B(\tau))}{S_r \geq S_R} \, \Probx{x}{S_r \geq S_R} \nonumber \\
            &=& \Expx{x}{\phi(B(S_r))}\, \Probx{x}{S_r < S_R} +
                \Expx{x}{\phi(B(S_R))}\, \Probx{x}{S_r \geq S_R} \nonumber \\
            &=& \phi(B(S_r))\, \Probx{x}{S_r < S_R} +
                \phi(B(S_R))\, \Probx{x}{S_r \geq S_R} \nonumber \\
            &=& \log(\abs{B(S_r)})\, \Probx{x}{S_r < S_R} +
                \log(\abs{B(S_R)})\, \Probx{x}{S_r \geq S_R} \nonumber \\
            &=& \log(r)\, \Probx{x}{S_r < S_R} +
                \log(R)\, \Probx{x}{S_r \geq S_R} \nonumber \\
            &=& \log(r)\, \Probx{x}{S_r < S_R} +
                \log(R)\, (1 - \Probx{x}{S_r < S_R}) \nonumber \\
            &=& \Probx{x}{S_r < S_R} (\log(r) - \log(R)) + \log(R) \nonumber \\
            &=& \Probx{x}{S_r < S_R} (\phi(r) - \phi(R)) + \phi(R) \label{eq:E}
    \end{eqnarray}
    Ahora bien, el ítem anterior nos asegura que $\Expx{x}{\phi(B(\tau))} = (\ref{eq:E}) = \phi(x)$,
    por lo que
    $$\Probx{x}{S_r < S_R} = \frac{\phi(x) - \phi(R)}{\phi(r) - \phi(R)} 
        = \frac{\phi(R) - \phi(x)}{\phi(R) - \phi(r)}$$

    \item Al considerar el evento $A = \Ev{S_r < \infty}$, observamos que
    $$\bigcup_{n \, \in \, \mathbb{N}}{\Ev{S_r < S_n}} \subseteq A$$
    Además, si $n < m$, se tiene que $\Ev{S_r < S_n} \subseteq \Ev{S_r < S_m}$. Por todo esto,
    \begin{eqnarray*}
        \Prob{S_r < \infty} &\geq& \Prob{\bigcup_{n \, \in \, \mathbb{N}}{\Ev{S_r < S_n}}} \\
            &=& \lim_{n \to \infty}{\Prob{S_r < S_n}} \\
            &=& \lim_{n \to \infty}{\frac{\phi(n) - \phi(x)}{\phi(n) - \phi(r)}} \\
            &=& 1
    \end{eqnarray*}
    En consecuencia, $\Prob{S_r < \infty} = 1$. Por otro lado, dado un abierto arbitrario
    $G \subset \mathbb{R}^2$, sabemos que $G$ puede expresarse como una unión numerable
    de bolas abiertas $U_1,\dots,U_n,\dots$.
    Por lo visto más arriba, tenemos que $\Prob{S_{u - \epsilon} < \infty} = 1$,
    siendo $u$ el radio de $U_1$ y $0 < \epsilon < u$. Notemos $\tilde{B}$ al movimiento
    browniano planar trasladado al centro de $U_1$. Al ser $S_{u - \epsilon}$ un
    tiempo de parada, por la propiedad de Markov fuerte de $\tilde{B}$ se tiene que el proceso
    $(\tilde{B}(S_{u-\epsilon} + t) - \tilde{B}(S_{u-\epsilon}), t \geq 0)$
    es un movimiento browniano independiente de $\mathcal{F}^{+}(S_{u-\epsilon})$,
    por lo que una repetición de este mismo razonamiento nos lleva a concluir que 
    $$\Prob{\tilde{B} \in U_1 \textrm{ infinitas veces}} = 1$$
    Pero $U_1 \subseteq G$, de lo que se desprende que
    $$\Prob{\tilde{B} \in G \textrm{ infinitas veces}} = 1$$

    \item Sea $G \subset \mathbb{R}^2$ un abierto y $E$ el evento ``el movimiento browniano planar
    $B$ entra en $G$ infinitas veces''. Con el objeto de formalizar $E$, nos interesaría expresar
    que, dado cualquier tiempo $t \in \mathbb{R}$, es posible encontrar un tiempo $t' > t$ tal que
    $B(t') \in G$. En primera instancia, notaremos $T_0$ al primer instante en el que $B$ entra
    en $G$:
    $$T_0 = \inf \left\{ t > 0 : B(t) \in G \right\}$$
    Definamos ahora los instantes $T_s$ en los que $B$ regresa a $G$, con $s > 0$:
    $$T_s = \sup \left\{ T_0 < t < T_0+s : B(t) \in G \right\}$$
    En función de estos tiempos aleatorios, podemos escribir a $E$ de la siguiente forma:
    $$E = \bigcap_{q \, \in \, \mathbb{Q}}{\bigcup_{k \, \in \, \mathbb{N}}{\Ev{T_k > q}}}$$

    \item Observemos primero que
    $$S_0 = \inf \left\{ t > 0 : \abs{B(t)} = 0 \right\}
          = \inf \left\{ t > 0 : B(t) = 0 \right\}$$
    En otras palabras, nos interesa probar que $B$ no pasa por el origen casi seguramente. Esto 
    en definitiva es una consecuencia directa del Corolario 2.26 del libro, estudiado también en 
    clase. Éste asegura que, dados cualquier par de puntos $x, y \in \mathbb{R}^2$,
    $\Probx{x}{y \in B(0,1]} = 0$. En este caso, instanciar el resultado con $x \neq 0$ e $y = 0$
    nos permite afirmar que $B(t)$ casi seguramente no pasa por el origen cuando
    $0 \leq t \leq 1$. No obstante, un razonamiento inductivo mediante este corolario junto con la
    propiedad de Markov de $B$ nos lleva a concluir que $\Probx{x}{S_0 < \infty} = 0$.

    \item Cuando $x = 0$ también podemos utilizar el corolario mencionado anteriormente para
    afirmar que $B(t)$ casi seguramente no pasa por el origen cuando $0 < t \leq 1$. El resto
    del razonamiento es análogo. Notar que $S_0$ está definido como el ínfimo de los tiempos
    positivos en los que $B = 0$ que, como consecuencia de lo mencionado, es casi seguramente
    un conjunto vacío. De esta forma, $\Probx{x}{S_0 = \infty} = 1$, i.e., $\Probx{x}{S_0 < \infty} = 0$.
\end{enumerate}
\end{proof}

\end{document}
