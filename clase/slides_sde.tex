\documentclass[10pt]{beamer}

\usetheme[progressbar=frametitle]{metropolis}

\usepackage{booktabs}
\usepackage[scale=2]{ccicons}

\usepackage{pgfplots}
\usepackage{graphicx}
\usepackage{tikz}
\usepackage{caption}
\usepackage{fancyvrb}
\usepackage{amsmath}
\usepackage{systeme}
\usepackage[spanish]{babel}
\usepackage[utf8]{inputenc}
\usepackage[export]{adjustbox}
\usepackage{xcolor}
\usepackage{color}
\usepgfplotslibrary{dateplot}

\usepackage{xspace}


\newcommand{\at}{@}
\newcommand\encircle[1]{%
  \tikz[baseline=(X.base)] 
    \node (X) [draw, shape=circle, inner sep=0] {\strut #1};}

\setbeamercolor{blockbg}{fg=normal text.fg,bg=normal text.bg!90!fg}

\definecolor{amber}{rgb}{1.0, 0.75, 0.0}
\definecolor{burntorange}{rgb}{0.8, 0.33, 0.0}

\newenvironment{colorblock}[1]{%
  \setbeamercolor{block title}{fg=normal text.fg,bg=burntorange!40}
  \setbeamercolor{block body}{fg=normal text.fg,bg=burntorange!15}
  \begin{block}{#1}}{\end{block}}

\title{Simulación Numérica de \\ Ecuaciones Diferenciales Estocásticas}
\date{3 de julio de 2017}
\author{Lucio Santi - \texttt{lsanti\at dc.uba.ar}}
%\institute{Universidad de Buenos Aires}
\titlegraphic{%
  %\mbox{}%
%   \vspace{-0.4cm} 
 %  \tikz\node[opacity=0.2] {\includegraphics[height=3.5cm]{images/fermilab_logo.png}};
   
  %\hspace{3cm} \quad
    %\vspace{-9.9cm}
    \vspace{-3.65cm} 
    \hspace{5.5cm} 
     \tikz\node[opacity=0.13] {\includegraphics[height=5.8cm]{images/uba_logo.jpg}};
}
%\titlegraphic{\hfill\includegraphics[height=1.5cm]{logo}}
\institute[shortinst]{Facultad de Ciencias Exactas y Naturales \\ %
                      Universidad de Buenos Aires
}

\setbeamertemplate{frame footer}{Simulación numérica de EDSs - Lucio Santi}
\metroset{block=fill}
\setbeamertemplate{itemize subitem}[triangle]

\newcommand{\Real}{\mathbb{R}}
\newcommand{\Dif}[1]{d #1}
\newcommand{\Intt}[4]{\int_{#1}^{#2}{#3~\Dif{#4}}}
\newcommand{\IntB}[1]{\int_0^t{#1~\Dif{B}}}
\newcommand{\IntBu}[1]{\int_0^t{#1~\Dif{B(u)}}}


\begin{document}

\maketitle

\begin{frame}{Agenda}
  \setbeamertemplate{section in toc}[circle]
  \setbeamertemplate{subsection in toc}[ball unnumbered]
  %\\tableofcontents[hideallsubsections]
   \tableofcontents[subsubsectionstyle=hide]
\end{frame}

\section{Introducción}

\subsection{Motivación}

\frame
{
    \frametitle{Motivación}

    \begin{itemize}
        \item Las \textbf{ecuaciones diferenciales estocásticas} (EDSs) son ecuaciones diferenciales 
        en las que al menos uno de sus términos es un proceso estocástico.
        \item Surgen como mecanismo de modelado
        en contextos diversos (e.g., biología, astronomía, finanzas y otros).
        \item Típicamente es difícil (sino imposible) encontrar soluciones analíticas.
        \item Por ende, es fundamental la aproximación de soluciones mediante \textbf{simulación numérica}.
        \item En esta exposición vamos a estudiar dos métodos numéricos de resolución de EDSs y algunas 
        de sus propiedades teóricas.
    \end{itemize}   
}

\subsection{Definiciones}

\frame
{
    \frametitle{Ecuación diferencial estocástica}

    %  X(t) = \onslide<2-> \underbrace{ \onslide<1-> X(0) \onslide<2-> }_{\text{valor inicial}} \onslide<1->
    \begin{block}{Proceso de Itô (unidimensional) $X = \{X(t), t \geq 0\}$}
    \begin{displaymath}
        X(t) = X(0) + \Intt{0}{t}{a(s,X(s))}{s} + \Intt{0}{t}{b(s,X(s))}{B(s)}
    \end{displaymath}
    \end{block}
    \vspace{-0.1cm}
    \begin{itemize}
        \item<2-> $X(0)$ es el valor inicial; puede ser aleatorio.
        \item<3-> $a : \Real^2 \to \Real$ es el coeficiente de deriva (\textsl{drift}).
        \item<4-> $b : \Real^2 \to \Real$ es el coeficiente de difusión.
        \item<5-> $B = \{B(t), t \geq 0\}$ es el movimiento browniano lineal conductor de $X$.
    \end{itemize}

    \uncover<6->
    {
    \begin{block}{Ecuación diferencial estocástica (de Itô)}
    \begin{displaymath}
        \Dif{X(t)} = a(t,X(t)) ~\Dif{t} + b(t,X(t)) ~\Dif{B(t)}
    \end{displaymath}
    \end{block}
    }
}

\frame
{
    \frametitle{Aproximaciones de tiempo discreto}

    \begin{itemize}
        \item Existen distintas estrategias de resolución numérica de EDSs.
        \item La más eficiente y generalmente aplicable consiste en simular
        caminos muestrales (\textsl{sample paths}) mediante \textbf{aproximaciones 
        de tiempo discreto}.
    \end{itemize}

    \uncover<2->
    {
        \begin{block}{Discretización temporal}
        Una \textbf{discretización} del intervalo $[t_0, T]$ viene dada por 
        $N$ instantes
        $$t_0 = \tau_0 < \tau_1 < \dots < \tau_N = T$$
        \end{block}
    }

        \begin{itemize}
            \item<3-> $h_n = \tau_{n+1} - \tau_n$ es el $n$-ésimo incremento.
            \item<4-> En las discretizaciones de tiempo equidistante, se toma $h = h_n = \frac{T - t_0}{N}$
            con $N$ suficientemente grande para que $0 < h < 1$.
            \uncover<5->{$\Rightarrow$ $\tau_n = t_0 + nh$}
        \end{itemize}
}


%%%%

\section{Métodos de resolución}

\subsection{Generalidades}

% definiciones generales
% estabilidad y convergencia

\subsection{Método de Euler-Maruyama}

% repaso euler determinístico
% definición
% propiedades
% alguna prueba de estabilidad/convergencia?

\subsection{Método de Milstein}

% derivación (expansión ito-taylor) => definición
% propiedades


%%%% 

\section{Implementación y validación}

% demo en vivo

\subsection{Caso de estudio}

\subsection{Resultados}

%%%% 

\section{Conclusiones}



%----------------------------
\begin{frame}[standout]
  Gracias!\\
  
  Preguntas?
\end{frame}

\nocite{*}
\begin{frame}[allowframebreaks]{Referencias}
  \bibliography{slides_sde}
  \bibliographystyle{abbrv}
\end{frame}

\end{document}
