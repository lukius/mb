\documentclass[10pt]{beamer}

\usetheme[progressbar=frametitle]{metropolis}

\usepackage{booktabs}
\usepackage[scale=2]{ccicons}

\usepackage{pgfplots}
\usepackage{graphicx}
\usepackage{tikz}
\usepackage{caption}
\usepackage{fancyvrb}
\usepackage{amsmath}
\usepackage{systeme}
\usepackage[spanish]{babel}
\usepackage[utf8]{inputenc}
\usepackage[export]{adjustbox}
\usepgfplotslibrary{dateplot}

\usepackage{xspace}


\newcommand{\at}{@}
\newcommand\encircle[1]{%
  \tikz[baseline=(X.base)] 
    \node (X) [draw, shape=circle, inner sep=0] {\strut #1};}

\setbeamercolor{blockbg}{fg=normal text.fg,bg=normal text.bg!90!fg}

\definecolor{amber}{rgb}{1.0, 0.75, 0.0}
\definecolor{burntorange}{rgb}{0.8, 0.33, 0.0}

\newenvironment{colorblock}[1]{%
  \setbeamercolor{block title}{fg=normal text.fg,bg=burntorange!40}
  \setbeamercolor{block body}{fg=normal text.fg,bg=burntorange!15}
  \begin{block}{#1}}{\end{block}}

\title{Simulación Numérica de \\ Ecuaciones Diferenciales Estocásticas}
\date{3 de julio de 2017}
\author{Lucio Santi - \texttt{lsanti\at dc.uba.ar}}
%\institute{Universidad de Buenos Aires}
\titlegraphic{%
  %\mbox{}%
%   \vspace{-0.4cm} 
 %  \tikz\node[opacity=0.2] {\includegraphics[height=3.5cm]{images/fermilab_logo.png}};
   
  %\hspace{3cm} \quad
    %\vspace{-9.9cm}
    \vspace{-3.65cm} 
    \hspace{5.5cm} 
     \tikz\node[opacity=0.13] {\includegraphics[height=5.8cm]{images/uba_logo.jpg}};
}
%\titlegraphic{\hfill\includegraphics[height=1.5cm]{logo}}
\institute[shortinst]{Facultad de Ciencias Exactas y Naturales \\ %
                      Universidad de Buenos Aires
}

\setbeamertemplate{frame footer}{Simulación numérica de EDSs - Lucio Santi}
% \metroset{block=fill}
\setbeamertemplate{itemize subitem}[triangle]

\begin{document}

\maketitle

\begin{frame}{Agenda}
  \setbeamertemplate{section in toc}[circle]
  \setbeamertemplate{subsection in toc}[ball unnumbered]
  %\\tableofcontents[hideallsubsections]
   \tableofcontents[subsubsectionstyle=hide]
\end{frame}


\section{Introducción}

\subsection{Motivación}

\subsection{Definiciones}

%%%%

\section{Métodos de resolución}

\subsection{Generalidades}

% definiciones generales
% estabilidad y convergencia

\subsection{Método de Euler-Maruyama}

% repaso euler determinístico
% definición
% propiedades
% alguna prueba de estabilidad/convergencia?

\subsection{Método de Milstein}

% derivación (expansión ito-taylor) => definición
% propiedades


%%%% 

\section{Implementación y validación}

% demo en vivo

\subsection{Caso de estudio}

\subsection{Resultados}

%%%% 

\section{Conclusiones}



%----------------------------
\begin{frame}[standout]
  Gracias!\\
  
  Preguntas?
\end{frame}

\nocite{*}
\begin{frame}[allowframebreaks]{Referencias}
  \bibliography{slides_sde}
  \bibliographystyle{abbrv}
\end{frame}

\end{document}
