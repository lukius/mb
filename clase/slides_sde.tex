\documentclass[10pt]{beamer}

\usetheme[progressbar=frametitle]{metropolis}

\usepackage{booktabs}
\usepackage[scale=2]{ccicons}

\usepackage{pgfplots}
\usepackage{graphicx}
\usepackage{tikz}
\usepackage{caption}
\usepackage{fancyvrb}
\usepackage{amsmath}
\usepackage{systeme}
\usepackage[spanish]{babel}
\usepackage[utf8]{inputenc}
\usepackage[export]{adjustbox}
\usepackage{xcolor}
\usepackage{color}
\usepackage{xfrac}
\usepgfplotslibrary{dateplot}

\usepackage{xspace}


\newcommand{\at}{@}
\newcommand\encircle[1]{%
  \tikz[baseline=(X.base)] 
    \node (X) [draw, shape=circle, inner sep=0] {\strut #1};}

\setbeamercolor{blockbg}{fg=normal text.fg,bg=normal text.bg!90!fg}

\definecolor{amber}{rgb}{1.0, 0.75, 0.0}
\definecolor{burntorange}{rgb}{0.8, 0.33, 0.0}

\newenvironment{colorblock}[1]{%
  \setbeamercolor{block title}{fg=normal text.fg,bg=burntorange!40}
  \setbeamercolor{block body}{fg=normal text.fg,bg=burntorange!15}
  \begin{block}{#1}}{\end{block}}


\title{Simulación Numérica de \\ Ecuaciones Diferenciales Estocásticas}
\date{3 de julio de 2017}
\author{Lucio Santi - \texttt{lsanti\at dc.uba.ar}}
%\institute{Universidad de Buenos Aires}
\titlegraphic{%
  %\mbox{}%
%   \vspace{-0.4cm} 
 %  \tikz\node[opacity=0.2] {\includegraphics[height=3.5cm]{images/fermilab_logo.png}};
   
  %\hspace{3cm} \quad
    %\vspace{-9.9cm}
    \vspace{-3.65cm} 
    \hspace{5.5cm} 
     \tikz\node[opacity=0.13] {\includegraphics[height=5.8cm]{images/uba_logo.jpg}};
}
%\titlegraphic{\hfill\includegraphics[height=1.5cm]{logo}}
\institute[shortinst]{Facultad de Ciencias Exactas y Naturales \\ %
                      Universidad de Buenos Aires
}

\setbeamertemplate{frame footer}{Simulación numérica de EDSs - Lucio Santi}
\metroset{block=fill}
\setbeamertemplate{itemize subitem}[triangle]

\newcommand{\abs}[1]{\ensuremath{\left\lvert #1 \right\rvert}}
\newcommand{\Real}{\mathbb{R}}
\newcommand{\Dif}[1]{d #1}
\newcommand{\Intt}[4]{\int_{#1}^{#2}{#3~\Dif{#4}}}
\newcommand{\IntB}[1]{\int_0^t{#1~\Dif{B}}}
\newcommand{\IntBu}[1]{\int_0^t{#1~\Dif{B(u)}}}
\newcommand{\Der}[2]{\frac{\partial}{\partial #2}~{#1}}
\newcommand{\DerT}[2]{\frac{\partial^2}{\partial {#2}^2}~{#1}}
\newcommand{\Exp}[1]{\ensuremath{\mathbb{E} #1}}
\newcommand{\ExpT}[1]{\ensuremath{\mathbb{E}^2 #1}}


\begin{document}

\maketitle

\begin{frame}{Agenda}
  \setbeamertemplate{section in toc}[circle]
  \setbeamertemplate{subsection in toc}[ball unnumbered]
  %\\tableofcontents[hideallsubsections]
   \tableofcontents[subsubsectionstyle=hide]
\end{frame}

\section{Introducción}

\subsection{Motivación}

\frame
{
    \frametitle{Motivación}

    \begin{itemize}
        \item Las \textbf{ecuaciones diferenciales estocásticas} (EDSs) son ecuaciones diferenciales 
        en las que al menos uno de sus términos es un proceso estocástico.
        \item Surgen como mecanismo de modelado
        en contextos diversos (e.g., biología, astronomía, finanzas y otros).
        \item Típicamente es difícil (sino imposible) encontrar soluciones analíticas.
        \item Por ende, es fundamental la aproximación de soluciones mediante \textbf{simulación numérica}.
        \item En esta exposición vamos a estudiar dos métodos numéricos de resolución de EDSs y algunas 
        de sus propiedades teóricas.
    \end{itemize}   
}

\subsection{Definiciones}

\frame
{
    \frametitle{Ecuación diferencial estocástica}

    %  X(t) = \onslide<2-> \underbrace{ \onslide<1-> X(0) \onslide<2-> }_{\text{valor inicial}} \onslide<1->
    \begin{block}{Proceso de Itô (unidimensional) $X = \{X(t), t \geq 0\}$}
    \begin{displaymath}
        X(t) = X(0) + \Intt{0}{t}{a(s,X(s))}{s} + \Intt{0}{t}{b(s,X(s))}{B(s)}
    \end{displaymath}
    \end{block}
    \vspace{-0.1cm}
    \begin{itemize}
        \item<2-> $X(0)$ es el valor inicial; puede ser aleatorio.
        \item<3-> $a : \Real^2 \to \Real$ es el coeficiente de deriva (\textsl{drift}).
        \item<4-> $b : \Real^2 \to \Real$ es el coeficiente de difusión.
        \item<5-> $B = \{B(t), t \geq 0\}$ es el movimiento browniano lineal conductor de $X$.
    \end{itemize}

    \uncover<6->
    {
    \begin{block}{Ecuación diferencial estocástica (de Itô)}
    \begin{displaymath}
        \Dif{X(t)} = a(t,X(t)) ~\Dif{t} + b(t,X(t)) ~\Dif{B(t)}
    \end{displaymath}
    \end{block}
    }
}

\frame
{
    \frametitle{Discretización temporal}

    \begin{itemize}
        \item Existen distintas estrategias de resolución numérica de EDSs.
        \item La más eficiente y generalmente aplicable consiste en simular
        caminos muestrales (\textsl{sample paths}) mediante \textbf{aproximaciones 
        de tiempo discreto}.
    \end{itemize}

    \uncover<2->
    {
        \begin{block}{Discretización temporal}
        Una \textbf{discretización} del intervalo $[t_0, T]$ viene dada por 
        $N+1$ instantes
        $$t_0 = \tau_0 < \tau_1 < \dots < \tau_N = T$$
        \end{block}
    }

        \begin{itemize}
            \item<3-> $h_n = \tau_{n+1} - \tau_n$ es el $n$-ésimo incremento (o paso).
            \item<4-> En las discretizaciones de tiempo equidistante, se toma $h = h_n = \frac{T - t_0}{N}$
            con $N$ suficientemente grande para que $0 < h < 1$.
            \uncover<5->{$\Rightarrow$ $\tau_n = t_0 + nh$}
        \end{itemize}
}

\frame
{
    \frametitle{Aproximaciones de tiempo discreto}

    \begin{block}{Aproximación de tiempo discreto}
    Dada una discretización temporal como antes, una \textbf{aproximación de tiempo discreto}
    del proceso de Itô $\{X(t), t_0 \leq t \leq T\}$ es un proceso continuo
    $\{X^h(t), t_0 \leq t \leq T \}$ evaluado en los instantes $\tau_n$ e interpolado
    linealmente entre ellos.
    \end{block}    

    \begin{itemize}
        \item<2-> Lógicamente, se busca que $X^h_n = X^h(\tau_n)$ aproxime el valor de $X(\tau_n)$.
        \item<3-> En lo que sigue veremos cómo definir esquemas numéricos para este
        propósito.
    \end{itemize}
}

%%%%

\section{Métodos de resolución}

\subsection{Generalidades}

\frame
{
    \frametitle{Derivación de esquemas numéricos}

    \begin{itemize}
        \item Los métodos numéricos para resolver EDSs siguen una dinámica similar a los
        métodos tradicionales para resolver EDOs.
        \item A modo de (breve) repaso, dada una EDO
        $$\dot{x}(t) = f(t, x(t))$$
        el método más sencillo de resolución es el \textbf{método de Euler}.
        \item<2-> Si consideramos la expansión de Taylor de $x$ alrededor de $t$,
        \begin{eqnarray*}
            x(t + h) &=& x(t) + \dot{x}(t) ~h + \frac{\ddot{x}(t)}{2} h^2 + \dots \\
                    \uncover<3->{&\approx& x(t) + f(t, x(t)) ~h}
        \end{eqnarray*}
        \item<4-> Luego, dada una discretización temporal $\tau_0,\dots,\tau_N$ de paso $h$,
        el método de Euler viene dado por
        $$x^h_{n+1} = x^h_n + f(\tau_n, x^h_n) ~ h$$
    \end{itemize}
}

\frame
{
    \frametitle{Expansión de Itô - Taylor}

    \begin{itemize}
        \item En EDSs la derivación de los métodos viene dada por la \textbf{expansión de Itô - Taylor}.
        \item<2-> Dado un proceso de Itô $\{X(t), t_0 \leq t \leq T\}$ y $f$ dos veces continuamente diferenciable,
        por la fórmula de Itô se tiene
        \uncover<2->{
        \begin{eqnarray*}
        f(X(t)) &=& f(X(t_0))  \\ \\
        &+& \Intt{t_0}{t}{\onslide<3-> \underbrace{ \onslide<2-> \left[a(X(s))~ f'(X(s)) + \frac{1}{2}b^2(X(s))~ f''(X(s)) \right] \onslide<3-> }_{L^0 f} \onslide<2->}{s} \\
        &+& \Intt{t_0}{t}{\onslide<3-> \underbrace{ \onslide<2-> b(X(s)) ~f'(X(s)) \onslide<3-> }_{L^1 f} \onslide<2->}{B(s)} 
        \end{eqnarray*}
        }
    \end{itemize}
}

\frame[plain]
{
    \frametitle{Expansión de Itô - Taylor (cont.)}

    Suponiendo que $a$ y $b$ son dos veces continuamente diferenciables,
    por la aplicación de la fórmula de Itô sobre ambas,\vspace*{0.8cm}
    
    \begin{beamercolorbox}[wd=1.02\textwidth,ht=2.7cm,rounded=true]{blockbg}
    \small{
    \begin{eqnarray*}
        X(t) &=& X(t_0) \\
             &+& \Intt{t_0}{t}{\left[a(X(t_0)) + \Intt{t_0}{s}{L^0 a(X(u))}{u} + \Intt{t_0}{s}{L^1 a(X(u))}{B(u)} \right]}{s} \\
             &+& \Intt{t_0}{t}{\left[b(X(t_0)) + \Intt{t_0}{s}{L^0 b(X(u))}{u} + \Intt{t_0}{s}{L^1 b(X(u))}{B(u)} \right]}{B(s)}
    \end{eqnarray*}}
    \end{beamercolorbox}
}

\frame
{
    \frametitle{Expansión de Itô - Taylor (cont.)}

    Luego, \vspace*{0.2cm}

    \begin{beamercolorbox}[wd=1.02\textwidth,ht=0.7cm,rounded=true]{blockbg}
    $$X(t) = X(t_0) + a(X(t_0)) ~h + b(X(t_0)) ~\Delta B + R(t)$$
    \end{beamercolorbox}

    donde
    \begin{itemize}
        \item $h = t - t_0$,
        \item $\Delta B = B(t) - B(t_0)$, y
        \item $R(t)$ es el \textbf{resto}, compuesto por las integrales dobles que surgen al distribuir las integrales
        externas sobre sus respectivas sumas. 
    \end{itemize}

    Los esquemas numéricos para resolver EDSs surgen del truncamiento de esta expansión en distintos
    puntos, aplicando sucesivas veces la fórmula de Itô sobre $a$, $b$, $L^0 a$, $L^0 b$, $L^1 a$, $\dots$ 
    (suponiendo que $a$ y $b$ son suficientemente suaves). 
}

\frame
{
    \frametitle{Propiedades de un esquema numérico}

     Para que un esquema numérico resulte útil, es deseable que sea

    \begin{itemize}
        \item<2-> \textbf{Convergente}: la solución computada numéricamente
        se aproxima a la solución exacta cuando el paso $h$ tiende a cero.

        \item<3-> \textbf{Consistente}: el error de truncamiento tiende a cero
        con $h$ (es una condición necesaria para la convergencia).

        \item<4-> \textbf{Numéricamente estable}: los errores de propagación
        se mantienen acotados.
    \end{itemize}
}

\frame
{
    \frametitle{Convergencia en EDSs}

    En el caso estocástico se distinguen dos tipos de convergencia, 
    dependiendo del escenario de aplicación del método numérico:

    \begin{itemize}
        \item<2-> \textbf{Convergencia fuerte}, cuando se desea que 
        la aproximación de tiempo discreto $X^h$ provea una buena estimación de los caminos
        muestrales del proceso $X$. 

        \item<3-> \textbf{Convergencia débil}, cuando se busca que
        $X^h$ provea una buena estimación de la distribución de $X$.
    \end{itemize}
}

\frame
{
    \frametitle{Convergencia fuerte y débil}

    \uncover<1->
    {
        \begin{block}{Convergencia fuerte de orden $\beta$}
            La aproximación de tiempo discreto $X^h$ \textbf{converge fuertemente a $X$ con orden $\beta > 0$} 
            en tiempo $T$ si existe una constante $C > 0$ tal que
            $$\Exp{\abs{X(T) - X^h(T)}} \leq C \, h^{\beta}$$
        \end{block}
    }

    \uncover<2->
    {
        \begin{block}{Convergencia débil de orden $\beta$}
            La aproximación de tiempo discreto $X^h$ \textbf{converge débilmente a $X$ con orden $\beta > 0$} 
            en tiempo $T$ si existe una constante $C > 0$ tal que
            $$\Exp{\abs{g(X(T)) - g(X^h(T))}} \leq C \, h^{\beta}$$
            para toda función $g \in \mathcal{C}^{2(\beta+1)}_{p}$ ($g$ y sus $2(\beta+1)$ derivadas 
            tienen crecimiento polinomial)
        \end{block}
    }

}

\subsection{Método de Euler-Maruyama}

\frame
{
    \frametitle{Método de Euler-Maruyama}

    El \textbf{método de Euler-Maruyama} es el análogo del método de Euler para
    las EDSs.  Viene dado por el truncamiento de la expansión de Itô - Taylor
    luego de la primera iteración:

    \begin{eqnarray*}
        \uncover<2->{X(t) &=& X(t_0) + a(X(t_0)) ~h + b(X(t_0)) ~\Delta B + R(t) \\}
        \uncover<3->{    &\approx& X(t_0) + a(X(t_0)) ~h + b(X(t_0)) ~\Delta B}
    \end{eqnarray*}

    \uncover<4->
    {
        Luego, \vspace*{0.2cm}

        \begin{beamercolorbox}[wd=1.02\textwidth,ht=0.7cm,rounded=true]{blockbg}
        $$X^h_{n+1} = X^h_n + a(X^h_n) ~h + b(X^h_n) ~\Delta B^h_n$$
        \end{beamercolorbox}

        donde $\Delta B^h_n = B^h(\tau_{n+1}) - B^h(\tau_n)$ es el $n$-ésimo incremento
        de una aproximación de tiempo discreto del browniano $B$.
    }
}

\frame
{
    \frametitle{Euler-Maruyama: propiedades}

    El método de Euler-Maruyama tiene (al menos) las siguientes propiedades:

    \begin{itemize}
        \item<2-> Converge fuertemente con orden $\beta = \frac{1}{2}$
        cuando $a$ y $b$ son Lipschitz y poseen crecimiento lineal (hipótesis usuales para
        probar existencia y unicidad de soluciones fuertes).

        \item<3-> Converge débilmente con orden $\beta = 1$ cuando $a, b \in \mathcal{C}^4_p$.

        \item<4-> Es numéricamente estable bajo las hipótesis usuales de existencia y unicidad
        de soluciones fuertes.
    \end{itemize}

    \uncover<5->{En lo que sigue vamos a probar la convergencia fuerte del método.}
}

\frame
{
    \frametitle{Euler-Maruyama: convergencia fuerte}

    \begin{block}{Teorema}
    Sea $\{X(t), t_0 \leq t \leq T\}$ un proceso de Itô (autónomo) y $\{X^h(t), t_0 \leq t \leq T\}$
    la aproximación de tiempo discreto de paso $h = \frac{T-t_0}{N} < 1$ dada por el método de Euler-Maruyama.
    Si los coeficientes de deriva $a$ y difusión $b$ de $X$ son tales
    \begin{itemize}
        \item $\abs{a(x) - a(y)} \leq K \abs{x-y}$,
        \item $\abs{b(x) - b(y)} \leq K \abs{x-y}$, y
        \item $\abs{a(x)} + \abs{b(x)} \leq K (1 + \abs{x})$
    \end{itemize}
    para cierta constante $K$ y para cualquier $x, y \in \Real$, entonces
    existe una constante $C$ tal que
    $$\Exp{\abs{X(T) - X^h(T)}} \leq C \, h^{\sfrac{1}{2}}$$
    \end{block}
}

\frame[plain]
{
    \frametitle{Euler-Maruyama: convergencia fuerte (cont.)}

    \textbf{Demostración.} Sea
    $$\tilde{X}^h(t) = X(t_0) + \sum_{n : \tau_n \leq t}{\Intt{\tau_{n-1}}{\tau_n}{a(\tilde{X}^h(s))}{s} + \Intt{\tau_{n-1}}{\tau_n}{b(\tilde{X}^h(s))}{B(s)}}$$
    la extensión constante de a trozos de $\{X^h_n = X^h(\tau_n)\}$ (i.e., el proceso que vale $X^h_{n-1}$ para todo $t \in [\tau_{n-1}, \tau_n)$).
    \uncover<2-> {    
    Dado $t \in [\tau_n, \tau_{n+1})$,}
    \vspace{-0.05cm}
    \begin{eqnarray*}
        \uncover<2->{X(t) - \tilde{X}^h(t) &=& \left[X(t_0) + \Intt{t_0}{t}{a(X(s))}{s} + \Intt{t_0}{t}{b(X(s))}{B(s)} \right]  \\
            &-& \left[ X(t_0) + \sum_{n : \tau_n \leq t}{\Intt{\tau_{n-1}}{\tau_n}{a(\tilde{X}^h(s))}{s} + \Intt{\tau_{n-1}}{\tau_n}{b(\tilde{X}^h(s))}{B(s)}} \right]} \\
        \uncover<3->{&=& \onslide<4-> \underbrace{ \onslide<3-> \Intt{\tau_n}{t}{a(X(s))}{s} + \Intt{t_0}{\tau_n}{a(X(s)) - a(\tilde{X}^h(s))}{s} \onslide<4-> }_{a_1(t)} \onslide<3->  \\
                     &+& \onslide<4-> \underbrace{ \onslide<3-> \Intt{\tau_n}{t}{b(X(s))}{B(s)} + \Intt{t_0}{\tau_n}{b(X(s)) - b(\tilde{X}^h(s))}{B(s)} \onslide<4-> }_{b_1(t)} \onslide<3->  }
    \end{eqnarray*}
}

\frame[plain]
{
    \frametitle{Euler-Maruyama: convergencia fuerte (cont.)}

    Observemos que

    $$\Exp{\left[ a_1(t) + b_1(t) \right]^2} \leq 2 ~\Exp{a_1^2(t)} + 2 ~\Exp{b_1^2(t)}$$

    \uncover<2->
    {
    Entonces,

    \begin{eqnarray*}
        Z(t) = \sup_{s \leq t} ~\Exp{\left[X(s) - \tilde{X}^h(s) \right]^2} \uncover<3->{&=& \sup_{s \leq t} ~\Exp{\left[a_1(s) + b_1(s) \right]^2}}\\
        \uncover<4->{ &\leq& \sup_{s \leq t} ~ 2 ~\Exp{a_1^2(s)} + 2 ~\Exp{b_1^2(s)} } \\
        \uncover<5->{ &=& 2 ~ 
        \onslide<6-> \underbrace{ \onslide<5-> \sup_{s \leq t} ~ \Exp{a_1^2(s)} \onslide<6-> }_{A(t)} \onslide<5->  + 2 ~
        \onslide<6-> \underbrace{ \onslide<5-> \sup_{s \leq t} ~ \Exp{b_1^2(s)} \onslide<6-> }_{B(t)} \onslide<5-> \\ }
    \end{eqnarray*}
    }
}

\frame[plain]
{
    \frametitle{Euler-Maruyama: convergencia fuerte (cont.)}
    
    Además, para $s \in [\tau_n, \tau_{n+1})$, \vspace*{-0.5cm}
    
    \begin{eqnarray*}
        \Exp{a_1^2(s)} &\leq& 2~\Exp{\left[ \Intt{\tau_n}{s}{\abs{a(X(u))}}{u} \right]^2} + 2~\Exp{\left[ \Intt{t_0}{s}{\abs{a(X(u)) - a(\tilde{X}^h(u))}}{u} \right]^2} \\
        \uncover<2->{&\leq& 2K^2 ~\Exp{\left[ \Intt{\tau_n}{s}{1 + \abs{X(u)}}{u} \right]^2} + 2K^2 ~\Exp{\left[ \Intt{t_0}{s}{\abs{X(u) - \tilde{X}^h(u)}}{u} \right]^2}  } \\
        \uncover<3->{&\leq& 2K^2 ~\Exp{ \Intt{\tau_n}{s}{(1 + \abs{X(u)})^2}{u} } + 2K^2 Nh ~\Exp{ \Intt{t_0}{s}{\abs{X(u) - \tilde{X}^h(u)}^2}{u} }} \\
        \uncover<4->{&\leq& 4K^2(L+1)h + 2K^2  Nh ~ \Intt{t_0}{s}{Z(u)}{u}  } 
    \end{eqnarray*}

    \uncover<5->{
    donde el tercer paso sigue de la desigualdad de Cauchy-Schwarz y el cuarto de \vspace*{-0.5cm}

    \begin{eqnarray*}
        \Exp{\Intt{\tau_n}{s}{(1 + \abs{X(u)})^2}{u}} &\leq&  2~\Exp{\Intt{\tau_n}{s}{1}{u}} + 2~\Exp{\Intt{\tau_n}{s}{X^2(u)}{u}} \\
        &\leq& 2(L+1)h
    \end{eqnarray*}

    \vspace{-0.2cm}
    siendo $L = \sup_{u \leq T} ~\Exp{X^2(u)}$}
}

\frame[plain]
{
    \frametitle{Euler-Maruyama: convergencia fuerte (cont.)}

    Por la isometría de Itô, la misma cota es válida para $\Exp{b_1^2(s)}$. \uncover<2->{Entonces, \vspace{-0.4cm}}
    
    \begin{eqnarray*}
        \uncover<2->{Z(t) &\leq& 2A(t) + 2B(t)} \\
        \uncover<3->{&\leq& 16K^2(L+1)h + 8K^2  Nh ~ \Intt{t_0}{t}{Z(u)}{u}} \\
        \uncover<4->{&=& C_1 \, h + C_2 \Intt{t_0}{t}{Z(u)}{u}}
    \end{eqnarray*}
    \uncover<5-> {
    Por el Lema de Grönwall, entonces, \vspace{-0.3cm}
    $$Z(t) \leq (C_1 \, h) ~ e^{C_2 (t-t_0)} \leq C_3 \, h$$
    }
    \uncover<6-> {
    Pero \vspace{-0.4cm}}
    $$\uncover<6->{\ExpT{\abs{\tilde{X}^h(T) - X(T)}} \leq \Exp{\abs{\tilde{X}^h(T) - X(T)}^2}} \uncover<7->{\leq Z(T) } \uncover<8->{\leq C_3 \, h}$$
    \uncover<9->{
    Y finalmente,\vspace{-0.2cm}}
    $$\uncover<9->{\Exp{\abs{X^h(T) - X(T)}}} \uncover<10->{= \Exp{\abs{\tilde{X}^h(T) - X(T)}}} \uncover<11->{\leq C_4 \, h^{\sfrac{1}{2}}}$$

}

\subsection{Método de Milstein}

\frame
{
    \frametitle{Método de Milstein: derivación}

    Consideremos en la expansión de Itô - Taylor el término

    $$\Intt{t_0}{t}{\Intt{t_0}{s}{L^1 b(X(u))}{B(u)}}{B(s)} $$

    \uncover<2->
    {
    Aplicando la fórmula de Itô nuevamente,

    \small{
    \begin{eqnarray*}
        && \hspace{-1.3cm}
        \Intt{t_0}{t}{\Intt{t_0}{s}{ \left[ L^1 b(X(t_0)) + \Intt{t_0}{u}{L^0 L^1 b(X(z))}{z} + \Intt{t_0}{u}{L^1 L^1 b(X(z)))}{B(z)} \right] }{B(u)}}{B(s)} \\
        \uncover<3->{&=& L^1 b(X(t_0)) ~ \Intt{t_0}{t}{\Intt{t_0}{s}{}{B(u)}}{B(s)} + \tilde{R}(t)} \\
        \uncover<4->{&=& b(X(t_0))~ b'(X(t_0)) ~ \Intt{t_0}{t}{\Intt{t_0}{s}{}{B(u)}}{B(s)} + \tilde{R}(t)} \\
    \end{eqnarray*}}
    }
}

\frame
{
    \frametitle{Método de Milstein: derivación (cont.)}

    Desarrollemos ahora 

    \begin{eqnarray*}
        \Intt{t_0}{t}{\Intt{t_0}{s}{}{B(u)}}{B(s)} \uncover<2->{&=& \Intt{t_0}{t}{B(s) - B(t_0)}{B(s)}} \\
        \uncover<3->{&=& \Intt{t_0}{t}{B(s)}{B(s)} - B(t_0)(B(t) - B(t_0))} \\ 
        \uncover<4->{&=& \frac{B^2(t) - B^2(t_0)}{2} - \frac{t - t_0}{2} - B(t_0)(B(t) - B(t_0))} \\ 
        \uncover<5->{&=& \frac{\left(B(t) - B(t_0)\right)^2}{2} - \frac{t - t_0}{2}} \\ 
        \uncover<6->{&=& \sfrac{1}{2} ~ (\Delta B^2 - h)} \\ 
    \end{eqnarray*}
}

\frame
{
    \frametitle{Método de Milstein}

    En consecuencia, notando $X(t_0) = X_0$, \vspace{-0.5cm}

    \begin{eqnarray*}
        \uncover<2->{X(t) &=& X_0 + a(X_0) ~h + b(X_0) ~\Delta B + \sfrac{1}{2} b(X_0) b'(X_0) \left[\Delta B^2 - h\right] + \tilde{R}(t) \\}
        \uncover<3->{    &\approx& X_0 + a(X_0) ~h + b(X_0) ~\Delta B + \sfrac{1}{2} b(X_0) b'(X_0) \left[\Delta B^2 - h\right]}
    \end{eqnarray*}

    \uncover<4->
    {
        Esto da origen al \textbf{método de Milstein}, \vspace*{0.2cm}

        \begin{beamercolorbox}[wd=1.02\textwidth,ht=0.82cm,rounded=true]{blockbg}
        $$ \vspace{-0.2cm}
        X^h_{n+1} = X^h_n + a(X^h_n) ~h + b(X^h_n) ~\Delta B^h_n +  \sfrac{1}{2} b(X^h_n) b'(X^h_n) \left[\Delta {B^h_n}^2 - h\right]
        $$
        \end{beamercolorbox}

        donde $\Delta B^h_n = B^h(\tau_{n+1}) - B^h(\tau_n)$ es el $n$-ésimo incremento
        de una aproximación de tiempo discreto del browniano $B$.
    }
}

\frame
{
    \frametitle{Método de Milstein: propiedades}

    El método de Milstein tiene (al menos) las siguientes propiedades:

    \begin{itemize}
        \item<2-> Converge fuertemente con orden $\beta = 1$
        cuando $a, b$ son $\mathcal{C}^2$ (además de las hipótesis
        usuales de existencia y unicidad de soluciones fuertes).

        \item<3-> Converge débilmente con orden $\beta = 1$ cuando $a, b \in \mathcal{C}^4_p$.

    \end{itemize}
}

\frame
{
    \frametitle{Otros métodos}

    \begin{itemize}
        \item Existen muchos otros métodos con mejores órdenes de convergencia (fuerte y débil).
        \item Generalmente, las aproximaciones son más precisas al abarcar más
        términos de las integrales estocásticas en la expansión de Itô - Taylor.
        \item Éstas contienen más información de los caminos muestrales del movimiento browniano
        sobre los subintervalos de discretización.
        \item Existen también esquemas del estilo Runge-Kutta que evitan el cálculo de
        las derivadas de los coeficientes de deriva y difusión.
    \end{itemize}
}


%%%% 

\section{Implementación y validación}

% demo en vivo

\subsection{Caso de estudio}

\subsection{Resultados}

%%%% 

\section{Conclusiones}



%----------------------------
\begin{frame}[standout]
  Gracias!\\
  
  Preguntas?
\end{frame}

\nocite{*}
\begin{frame}[allowframebreaks]{Referencias}
  \bibliography{slides_sde}
  \bibliographystyle{abbrv}
\end{frame}

\end{document}
