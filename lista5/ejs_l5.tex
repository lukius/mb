\documentclass[a4paper,11pt]{article}
\usepackage[utf8x]{inputenc}
\usepackage{fancyhdr}
\usepackage[spanish]{babel}
\usepackage{lastpage}
\usepackage{amstext}
\usepackage{amsmath}
\usepackage{amsfonts}
\usepackage{amsthm}
\usepackage{amssymb}
\usepackage{enumerate}
\usepackage{graphicx}
\usepackage{etoolbox}
\usepackage[implicit=false]{hyperref}
\usepackage[a4paper, total={6.5in, 9.5in}]{geometry}
\usepackage[T1]{fontenc}
\usepackage[sc]{mathpazo}

\newcommand{\at}{@}

\title{Movimiento Browniano\\
      \small{Ejercicios entregables - Lista 5}}
\author{Lucio Santi\\
        \texttt{lsanti\at dc.uba.ar}}
\date{\today}

\pagestyle{fancyplain} 
\renewcommand{\headrulewidth}{0pt}
\cfoot{\thepage/\pageref{LastPage}}
\lhead{}
\chead{}
\rhead{}

\newcommand{\abs}[1]{\ensuremath{\left\lvert #1 \right\rvert}}
\newcommand{\Sig}[1]{\ensuremath{\mathcal{#1}}}
\newcommand{\SigAlg}[1]{\ensuremath{\sigma\left(#1\right)}}
\newcommand{\Mart}[2]{\ensuremath{\left(#1_n, \Sig{#2}_n\right)}}
\newcommand{\Exp}[1]{\ensuremath{\textrm{E}\left[#1\right]}}
\newcommand{\ExpC}[2]{\ensuremath{\textrm{E}\left[#1 \, | \, #2\right]}}
\newcommand{\Prob}[1]{\ensuremath{\mathbb{P} \left( #1 \right)}}
\newcommand{\Probx}[2]{\ensuremath{\mathbb{P}_{#1} \left( #2 \right)}}
\newcommand{\Expx}[2]{\ensuremath{\textrm{E}_{#1}\left[#2\right]}}
\newcommand{\ExpxC}[3]{\ensuremath{\textrm{E}_{#1}\left[#2 \, | \, #3\right]}}
\newcommand{\Ev}[1]{\ensuremath{\left\{ #1 \right\}}}
\newcommand{\norm}[1]{\left\lVert#1\right\rVert}
\newcommand{\normi}[1]{\norm{#1}_{\infty}}
\newcommand{\Sst}[1]{S_n^{*}(#1)}

\newtheorem*{ej}{Ejercicio}

\begin{document}
\maketitle

\begin{ej} 
Sean $X_1, X_2, \dots$ variables aleatorias i.i.d. con distribución $F$ y sea
$$\hat{F}_n(x) = \#\left\{m \leq n ∶  X_m \leq x \right\}$$
El objetivo de este ejercicio es probar que
$$D_n(x) = \sqrt{n} (\hat{F}_n(x) − F(x))$$
converge en distribución al \textsl{puente browniano}.

\begin{enumerate}[a.]
    \item Observar que, por la Ley de los Grandes Números,
    $\abs{\hat{F}_n(x) − F (x)} \to 0$ para todo $0 < x < 1$.

    \item Utilizando la transformación $Y_k = F(X_k)$ mostrar que basta
    considerar el caso en que $F$ es la distribución uniforme en $(0, 1)$.

    \item Sean $Y_1, Y_2, \dots$ i.i.d. $\mathcal{U}(0, 1)$ y
    $Y_{(1)} < Y_{(2)} < \dots$ la muestra ordenada. Sean
    $W_1, W_2, \dots$ i.i.d. $\mathcal{E}(1)$ y
    $Z_n = W_1 + \dots + W_n$. Probar que
    $(Y_{(1)}, \dots, Y_{(n)})$ y $(Z_1/Z_{n+1}, \dots, Z_n/Z_{n+1})$
    tienen la misma distribución. Sugerencia: hallar la densidad de
    $(Y_{(1)}, \dots, Y_{(n)})$ y de $r(Z_1, \dots, Z_{n+1})$, donde
    $$r(z_1, \dots, z_{n+1}) = (z_1/z_{n+1}, \dots, z_n/z_{n+1}, z_{n+1})$$

    \item Sea $\tilde{D}_{k,n} = \sqrt{n} (Z_k/Z_{n+1} − k/n)$ y extender al
    $[0, 1]$ interpolando linealmente. Probar que
    $$\normi{\tilde{D}_n - D_n} \to 0$$
    en probabilidad cuando $n \to \infty$.

    \item Sea $S_n = S_n(W_1 − 1, \dots, W_n − 1)$ y $S_n^{*}$ definidos como
    en la clase. Mostrar que
    $$\tilde{D}_n(x) = \frac{n}{Z_{n+1}} \left(\Sst{x} - x \left[ \Sst{1} + \frac{Z_{n+1} - Z_n}{\sqrt{n}} \right] \right)$$

    \item Probar que $Z_{n+1}/n \to 1$ y 
    $(Z_{n+1} - Z_n)/\sqrt{n} \to 0$
    en probabilidad.

    \item Asumir (o probar) que el Teorema de Slutsky vale también para
    sucesiones de variables aleatorias a valores en espacios métricos.

    \item Probar que los procesos $(D_n(x), 0 \leq x \leq 1)_{n \geq 1}$
    convergen al proceso $(W(t), 0 \leq t \leq 1)$ dado por
    $W(t) = B(t) − tB(1)$,
    denominado \textsl{puente browniano} ($B$ es un movimiento browniano).
\end{enumerate}
\end{ej}

\begin{proof}[Resoluci\'on]
$ $ 
\begin{enumerate}
    \item[d)]
\end{enumerate}
\end{proof}

\end{document}
