\documentclass[a4paper,11pt]{article}
\usepackage[utf8x]{inputenc}
\usepackage{fancyhdr}
\usepackage[spanish]{babel}
\usepackage{lastpage}
\usepackage{amstext}
\usepackage{amsmath}
\usepackage{amsfonts}
\usepackage{amsthm}
\usepackage{amssymb}
\usepackage{enumerate}
\usepackage{graphicx}
\usepackage{etoolbox}
\usepackage[implicit=false]{hyperref}
\usepackage[a4paper, total={6.5in, 9.5in}]{geometry}
\usepackage[T1]{fontenc}
\usepackage[sc]{mathpazo}

\newcommand{\at}{@}

\title{Movimiento Browniano\\
      \small{Ejercicios entregables - Lista 5}}
\author{Lucio Santi\\
        \texttt{lsanti\at dc.uba.ar}}
\date{\today}

\pagestyle{fancyplain} 
\renewcommand{\headrulewidth}{0pt}
\cfoot{\thepage/\pageref{LastPage}}
\lhead{}
\chead{}
\rhead{}

\newcommand{\abs}[1]{\ensuremath{\left\lvert #1 \right\rvert}}
\newcommand{\Sig}[1]{\ensuremath{\mathcal{#1}}}
\newcommand{\SigAlg}[1]{\ensuremath{\sigma\left(#1\right)}}
\newcommand{\Mart}[2]{\ensuremath{\left(#1_n, \Sig{#2}_n\right)}}
\newcommand{\Exp}[1]{\ensuremath{\textrm{E}\left[#1\right]}}
\newcommand{\ExpC}[2]{\ensuremath{\textrm{E}\left[#1 \, | \, #2\right]}}
\newcommand{\Prob}[1]{\ensuremath{\mathbb{P} \left( #1 \right)}}
\newcommand{\Probx}[2]{\ensuremath{\mathbb{P}_{#1} \left( #2 \right)}}
\newcommand{\Expx}[2]{\ensuremath{\textrm{E}_{#1}\left[#2\right]}}
\newcommand{\ExpxC}[3]{\ensuremath{\textrm{E}_{#1}\left[#2 \, | \, #3\right]}}
\newcommand{\Ev}[1]{\ensuremath{\left\{ #1 \right\}}}
\newcommand{\norm}[1]{\left\lVert#1\right\rVert}
\newcommand{\normi}[1]{\norm{#1}_{\infty}}
\newcommand{\Sst}[1]{S_n^{*}(#1)}
\newcommand{\floor}[1]{\left\lfloor #1 \right\rfloor}
\newcommand{\Dn}[1]{\tilde{D}_{#1 ,n}}
\newcommand{\fnx}{\floor{nx}}
\newcommand{\fnxp}{\floor{nx}+1}

\newtheorem*{ej}{Ejercicio}

\begin{document}
\maketitle

\begin{ej} 
Sean $X_1, X_2, \dots$ variables aleatorias i.i.d. con distribución $F$ y sea
$$\hat{F}_n(x) = \#\left\{m \leq n ∶  X_m \leq x \right\}$$
El objetivo de este ejercicio es probar que
$$D_n(x) = \sqrt{n} (\hat{F}_n(x) − F(x))$$
converge en distribución al \textsl{puente browniano}.

\begin{enumerate}[a.]
    \item Observar que, por la Ley de los Grandes Números,
    $\abs{\hat{F}_n(x) − F (x)} \to 0$ para todo $0 < x < 1$.

    \item Utilizando la transformación $Y_k = F(X_k)$ mostrar que basta
    considerar el caso en que $F$ es la distribución uniforme en $(0, 1)$.

    \item Sean $Y_1, Y_2, \dots$ i.i.d. $\mathcal{U}(0, 1)$ y
    $Y_{(1)} < Y_{(2)} < \dots$ la muestra ordenada. Sean
    $W_1, W_2, \dots$ i.i.d. $\mathcal{E}(1)$ y
    $Z_n = W_1 + \dots + W_n$. Probar que
    $(Y_{(1)}, \dots, Y_{(n)})$ y $(Z_1/Z_{n+1}, \dots, Z_n/Z_{n+1})$
    tienen la misma distribución. Sugerencia: hallar la densidad de
    $(Y_{(1)}, \dots, Y_{(n)})$ y de $r(Z_1, \dots, Z_{n+1})$, donde
    $$r(z_1, \dots, z_{n+1}) = (z_1/z_{n+1}, \dots, z_n/z_{n+1}, z_{n+1})$$

    \item Sea $\tilde{D}_{k,n} = \sqrt{n} (Z_k/Z_{n+1} − k/n)$ y extender al
    $[0, 1]$ interpolando linealmente. Probar que
    $$\normi{\tilde{D}_n - D_n} \to 0$$
    en probabilidad cuando $n \to \infty$.

    \item Sea $S_n = S_n(W_1 − 1, \dots, W_n − 1)$ y $S_n^{*}$ definidos como
    en la clase. Mostrar que
    $$\tilde{D}_n(x) = \frac{n}{Z_{n+1}} \left(\Sst{x} - x \left[ \Sst{1} + \frac{Z_{n+1} - Z_n}{\sqrt{n}} \right] \right)$$

    \item Probar que $Z_{n+1}/n \to 1$ y 
    $(Z_{n+1} - Z_n)/\sqrt{n} \to 0$
    en probabilidad.

    \item Asumir (o probar) que el Teorema de Slutsky vale también para
    sucesiones de variables aleatorias a valores en espacios métricos.

    \item Probar que los procesos $(D_n(x), 0 \leq x \leq 1)_{n \geq 1}$
    convergen al proceso $(W(t), 0 \leq t \leq 1)$ dado por
    $W(t) = B(t) − tB(1)$,
    denominado \textsl{puente browniano} ($B$ es un movimiento browniano).
\end{enumerate}
\end{ej}

\begin{proof}[Resoluci\'on]
$ $ 
\begin{enumerate}
    \item[d.] TBD

    \item[e.] Formulemos primero la extensión de $\tilde{D}_n$ al intervalo $[0,1]$
    a partir de la definición de $\Dn{k}$, $0 \leq k \leq n$:
    $$\tilde{D}_n(x) = \Dn{\floor{nx}} + (nx - \floor{nx}) (\Dn{\floor{nx} + 1} - \Dn{\floor{nx}})$$
    Tenemos además
    $$S_n = \sum_{i = 1}^{n}{W_i - 1} = Z_n - n$$
    y, para $x \in \mathbb{R}_{\geq 0}$,
    $$S(x) = S_{\floor{x}} + (x - \floor{x}) (S_{\floor{x}+1} - S_{\floor{x}})$$
    Finalmente,
    $$\Sst{x} = \frac{S(nx)}{\sqrt{n}}$$

    Desarrollemos ahora $\tilde{D}_n$ a partir de todo lo anterior:
    \begin{eqnarray*}
        \tilde{D}_n(x)
            &=& \Dn{\floor{nx}} + (nx - \floor{nx}) \left(\Dn{\floor{nx} + 1} - \Dn{\floor{nx}}\right) \\
            &=& \sqrt{n} \, \left(\frac{Z_{\fnx}}{Z_{n+1}} − \frac{\fnx}{n} \right) + \sqrt{n} \, (nx - \floor{nx}) \\
            &\quad&  \left(\left(\frac{Z_{\fnxp}}{Z_{n+1}} − \frac{\fnxp}{n}\right) - \left(\frac{Z_{\fnx}}{Z_{n+1}} − \frac{\fnx}{n}\right)\right) \\
            &=& \sqrt{n} \, \left(\frac{Z_{\fnx}}{Z_{n+1}} − \frac{\fnx}{n} \right) + \frac{\sqrt{n}}{Z_{n+1}} \, (nx - \floor{nx}) \\
            &\quad&  \left(Z_{\fnxp} − Z_{n+1} \frac{\fnxp}{n} - Z_{\fnx} + Z_{n+1} \frac{\fnx}{n} \right) \\
            &=& \sqrt{n} \, \left(\frac{Z_{\fnx}}{Z_{n+1}} − \frac{\fnx}{n} \right) + \frac{\sqrt{n}}{Z_{n+1}} \, (nx - \floor{nx}) 
                \left(Z_{\fnxp} - Z_{\fnx} - \frac{Z_{n+1}}{n} \right) \\
            &=& A_n(x) + B_n(x)
    \end{eqnarray*}
    donde
    \begin{itemize}
        \item $A_n(x) = \frac{\sqrt{n}}{Z_{n+1}} \, (nx - \floor{nx})  \left(Z_{\fnxp} - Z_{\fnx} - 1 \right)$ y
        \item $B_n(x) =  \sqrt{n} \, \left(\frac{Z_{\fnx}}{Z_{n+1}} − \frac{\fnx}{n} \right) + \frac{\sqrt{n}}{Z_{n+1}} \, (nx - \floor{nx}) \left(1 - \frac{Z_{n+1}}{n}\right)$
    \end{itemize}

    Por otro lado, sea
    $$C_n(x) = \frac{n}{Z_{n+1}} \left(\Sst{x} - x \left[ \Sst{1} + \frac{Z_{n+1} - Z_n}{\sqrt{n}} \right] \right)$$

    Desarrollando $C_n$, tenemos

    \begin{eqnarray*}
        C_n(x) &=& \frac{n}{Z_{n+1}} \left(\frac{S(nx)}{\sqrt{n}} - x \left[ \frac{S(n)}{\sqrt{n}} + \frac{Z_{n+1} - Z_n}{\sqrt{n}} \right] \right) \\
            &=& \frac{n}{Z_{n+1}} \left(\frac{S_{\floor{nx}} + (nx - \floor{nx}) (S_{\floor{nx}+1} - S_{\floor{nx}})}{\sqrt{n}} - x \left[ \frac{Z_n - n}{\sqrt{n}} + \frac{Z_{n+1} - Z_n}{\sqrt{n}} \right] \right) \\
            &=& \frac{\sqrt{n}}{Z_{n+1}} \left( \left(S_{\floor{nx}} + (nx - \floor{nx}) (S_{\floor{nx}+1} - S_{\floor{nx}})\right) - x \left( Z_{n+1} - n \right) \right) \\
            &=& \frac{\sqrt{n}}{Z_{n+1}} \left( \left(Z_{\floor{nx}} - \floor{nx} + (nx - \floor{nx}) (Z_{\floor{nx}+1} - Z_{\floor{nx}} - 1)\right) - x \left( Z_{n+1} - n \right) \right) \\
            &=& A_n(x) + \underbrace{ \frac{\sqrt{n}}{Z_{n+1}} \left(Z_{\floor{nx}} - \floor{nx}  - x \left( Z_{n+1} - n \right)\right) }_{G_n(x)}
    \end{eqnarray*}

    Por ende, resta ver que $B_n(x) = G_n(x)$. Desarrollemos entonces $B_n$:

    \begin{eqnarray*}    
        B_n(x) &=& \sqrt{n} \, \left(\frac{Z_{\fnx}}{Z_{n+1}} − \frac{\fnx}{n} \right) + \frac{\sqrt{n}}{Z_{n+1}} \, (nx - \floor{nx}) \left(1 - \frac{Z_{n+1}}{n}\right) \\
            &=& \frac{\sqrt{n}}{Z_{n+1}} \left( Z_{\fnx} - Z_{n+1} \frac{\fnx}{n} + (nx - \floor{nx}) \left(1 - \frac{Z_{n+1}}{n}\right) \right) \\
            &=& \frac{\sqrt{n}}{Z_{n+1}} \left( Z_{\fnx} - Z_{n+1} \frac{\fnx}{n} + nx - x Z_{n+1} - \fnx + \fnx \frac{Z_{n+1}}{n} \right) \\
            &=& \frac{\sqrt{n}}{Z_{n+1}} \left( Z_{\fnx} + nx - x Z_{n+1} - \fnx \right) \\
            &=& \frac{\sqrt{n}}{Z_{n+1}} \left( Z_{\fnx} - \fnx - x  (Z_{n+1} - n) \right) \\
            &=& G_n(x)
    \end{eqnarray*}

    \item[f.] En primera instancia,
    $$\frac{Z_{n+1}}{n} = \underbrace{\frac{\sum_{i=1}^{n}{W_i}}{n}}_{X_n} + \underbrace{\frac{W_{n+1}}{n}}_{Y_n}$$
    Por la Ley de los Grandes Números,
    $$X_n \overset{P}{\longrightarrow} \Exp{W_1} = 1$$
    Además, dado $\epsilon > 0$,
    \begin{eqnarray*}
        \Prob{\abs{Y_n} \geq \epsilon} &=& \Prob{\frac{W_{n+1}}{n} \geq \epsilon} \\
            &=& \Prob{W_{n+1} \geq n \epsilon} \\
            &=& 1 - \Prob{W_{n+1} < n \epsilon} \\
            &=& 1 - \left(1 - e^{-n \epsilon}\right) \\
            &=& e^{-n \epsilon} \\
            &\underset{n \to \infty}{\longrightarrow}& 0
    \end{eqnarray*}
    De manera que
    $$Y_n \overset{P}{\longrightarrow} 0$$
    De esta forma,
    $$\frac{Z_{n+1}}{n} = X_n + Y_n \overset{P}{\longrightarrow} 1$$

    Por otra parte, observemos que $Z_{n+1} - Z_n = W_{n+1}$. Mediante un razonamiento análogo
    al hecho para $Y_n$ anteriormente se puede ver que 
    $$\frac{Z_{n+1} - Z_n}{\sqrt{n}} \overset{P}{\longrightarrow} 0$$

    \item[h.] 
    Si notamos $X_n = W_n - 1$, tenemos que $\Exp{X_n} = \Exp{W_n} - 1 = 0$ y
    \begin{eqnarray*}
       \text{Var}[X_n] &=& \Exp{X_n^2} - \Exp{X_n}^2 \\
            &=& \Exp{(W_n - 1)^2} \\
            &=& \Exp{W_n^2 - 2W_n + 1} \\
            &=& \Exp{W_n^2} - 2\Exp{W_n} + 1 \\
            &=& 1
    \end{eqnarray*}
    En consecuencia, siendo $S_n$ definida en términos de $X_1,\dots,X_n$, estamos en las hipótesis del teorema 
    de invariancia de Donsker. Éste nos permite concluir que
    $$S_n^{*} \overset{\mathcal{D}}{\longrightarrow} B$$
    en $C[0,1]$, siendo $B$ un movimiento browniano.

    Ahora bien, del ítem e. sabemos que 
    $$\tilde{D}_n(x) = \frac{n}{Z_{n+1}} \left(\Sst{x} - x \left[ \Sst{1} + \frac{Z_{n+1} - Z_n}{\sqrt{n}} \right] \right)$$
    Mediante la observación anterior y los resultados del ítem f. se tiene que
    $$\tilde{D}_n \overset{\mathcal{D}}{\longrightarrow} W$$
    siendo $W(x) = B(x) - xB(1)$ el puente browniano.

    Además, por el ítem d.,
    $$\normi{\tilde{D}_n - D_n} \overset{P}{\longrightarrow} 0$$
    con lo cual $D_n - \tilde{D}_n$ converge a la función constante 0 en probabilidad. Luego, por el teorema
    de Slutsky generalizado a variables aleatorias a valores en $(C[0,1], \normi{\cdot})$,
    $$D_n = \tilde{D}_n + (D_n - \tilde{D}_n) \overset{\mathcal{D}}{\longrightarrow} W + 0 = W$$
\end{enumerate}
\end{proof}

\end{document}
