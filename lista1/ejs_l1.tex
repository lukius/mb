\documentclass[a4paper,11pt]{article}
\usepackage[utf8x]{inputenc}
\usepackage{fancyhdr}
\usepackage[spanish]{babel}
\usepackage{lastpage}
\usepackage{amstext}
\usepackage{amsmath}
\usepackage{amsfonts}
\usepackage{amsthm}
\usepackage{amssymb}
\usepackage{enumerate}
\usepackage{graphicx}
\usepackage{etoolbox}
\usepackage[implicit=false]{hyperref}
\usepackage[a4paper, total={6.5in, 9.5in}]{geometry}
\usepackage[T1]{fontenc}
\usepackage[sc]{mathpazo}

\newcommand{\at}{@}

\title{Movimiento Browniano\\
      \small{Ejercicios entregables - Semana 1}}
\author{Lucio Santi\\
        \texttt{lsanti\at dc.uba.ar}}
\date{\today}

\pagestyle{fancyplain} 
\renewcommand{\headrulewidth}{0pt}
\cfoot{\thepage/\pageref{LastPage}}
\lhead{}
\chead{}
\rhead{}

\newcommand{\norm}[1]{\left\lVert#1\right\rVert}
\newcommand{\normi}[1]{\norm{#1}_{\infty}}
\newcommand{\CT}{\ensuremath{C([0,T], \mathbb{R})}}
\newcommand{\CTesp}{\ensuremath{(C([0,T], \mathbb{R}), \norm{\cdot}_{\infty})}}
\newcommand{\Bor}[1]{\ensuremath{\mathcal{B}(#1)}}
\newcommand{\floor}[1]{\ensuremath{\left\lfloor #1 \right\rfloor}}
\newcommand{\cupdot}{\mathbin{\mathaccent\cdot\cup}}
\newcommand{\Bola}[2]{\ensuremath{B_{#2}(#1)}}

\newtheorem*{ej}{Ejercicio}

\begin{document}
\maketitle

\begin{ej} 
    Este ejercicio es para caracterizar la $\sigma$-álgebra de Borel
    $\mathcal{B}$ en \CT.

    \begin{enumerate}[a)]
        \item Sea $(E,d)$ un espacio métrico separable y completo (polaco).
        Probar que todo abierto $U \subset E$ se puede escribir como unión
        numerable de bolas abiertas.
        
        \item Sea $(E,d)$ un espacio métrico polaco. Probar que existen 
        numerables bolas $B_1, \dots, B_n, \dots$ tal que la $\sigma$-álgebra
        de Borel \Bor{E} verifica
        $$\Bor{E} = \sigma\left(\{B_n : n \in \mathbb{N} \}\right)$$

        \item Para $\omega \in \CT$ definimos $\pi_t(\omega) = \omega(t)$.
        Probar que $\pi_t : \CT \rightarrow \mathbb{R}$ es continua.\\

        En \CTesp \, definimos la $\sigma$-álgebra de Kolmogorov,
        $$\mathcal{K} = \sigma\left(\left\{\pi_{t}^{-1}(B) : t \in [0,T],
        B \in \Bor{\mathbb{R}} \right\}\right)$$

        \item Probar que las bolas abiertas están en $\mathcal{K}$.

        \item Probar que $\mathcal{K} = \mathcal{B}$.
    \end{enumerate}

\end{ej}

\begin{proof}[Resoluci\'on]
$ $

\begin{enumerate}[a)]
    \item Sea $U \subset E$ un conjunto abierto no vacío. Por ser $(E,d)$ un
    espacio métrico separable, sabemos que existe
    $S \subset E$ numerable y denso, i.e., $S = \{s_1,\dots,s_n,\dots\}$
    es tal que $U \cap S = V \neq \emptyset$. Podemos entonces escribir
    $U = V \cupdot W$, donde $W$ es tal que no posee ningún subconjunto
    abierto (de lo contrario, un tal subconjunto $X$ satisfaría 
    $X \cap S \neq \emptyset$, de manera que $X \subset V$).
    Por ser $U$ abierto, para cada $v = s_i \in V$,
    se tiene que existe $\epsilon_i > 0$ tal que
    $\Bola{s_i}{\epsilon_i} \subset U$. Sea
    $\epsilon = \inf \left\{ \epsilon_n : n \in \mathbb{N} \right\}$.
    Dado $w \in W$, tenemos como antes que existe $\delta > 0$ tal que
    $\Bola{w}{\delta} \subset U$, de manera que debe existir por lo menos
    un $s_j \in V$ en $\Bola{w}{\delta}$. De no ser así,
    $\Bola{w}{\delta} \subset W$, pero ya argumentamos que $W$ no puede
    tener subconjuntos abiertos. Sea $\beta = \min(\delta, \epsilon)$\footnote
    {Si $\epsilon = 0$ hay que cambiar la estrategia.}.
    Luego, por este mismo razonamiento, $s_j \in \Bola{w}{\beta}$ para
    cierto $j \in \mathbb{N}$, y
    $d(s_j,w) = d(w,s_j) < \beta \leq \epsilon \leq \epsilon_j \Rightarrow 
    w \in \Bola{s_j}{\epsilon_j}$. Esto sugiere tomar en consideración las
    bolas
    $$B_n = \left\{ x \in E : d(s_n, x) < \epsilon_n \right\}$$
    para cada $n \in \mathbb{N}$. Por todo lo anterior, se observa que
    $U = \bigcup B_n$\footnote{Se ve claramente que mi argumento no utiliza
    la hipótesis de que $(E,d)$ es completo. ?`Es realmente necesaria?}.

    \item TBD

    \item Sea $\omega_0 \in \CT$, $\epsilon >0$ y definamos $\delta = \epsilon/2$.
    Supongamos que, para cierta $\omega \in \CT$,
    $\normi{\omega - \omega_0} < \delta$. Entonces,
    \begin{eqnarray*}
        |\pi_t(\omega) - \pi_t(\omega_0)| &=& |\omega(t) - \omega_0(t)| \\
        &\leq& \sup \left\{ |\omega(s) - \omega_0(s)| : s \in [0,T] \right\} \\
        &=& \normi{\omega - \omega_0} \\
        &<& \delta \\
        &<& \epsilon
    \end{eqnarray*}
    De esto sigue que $\pi_t$ es continua en cualquier $\omega_0$ y, por lo tanto,
    continua en todo su dominio.

    \item Sea $\omega \in \CT$ y \Bola{\omega}{\epsilon} una bola abierta.
    Consideremos
    $S_{\epsilon}(\omega) = \{ \omega_0(t) : \omega_0 \in \Bola{\omega}{\epsilon} \}$.
    Probar lo solicitado se reduce a probar que $S_{\epsilon}(\omega)$
    es abierto en $\mathbb{R}$: de ser así, $S_{\epsilon}(\omega) \in \Bor{\mathbb{R}}$,
    por lo que la $\sigma$-álgebra de Kolmogorov contendrá a \Bola{\omega}{\epsilon}.
    Sea entonces $x \in S_{\epsilon}(\omega)$. Esto implica que $x = \omega_1(t)$ para
    cierta $\omega_1 \in \Bola{\omega}{\epsilon}$. Sea
    $\delta = \epsilon - \normi{\omega - \omega_1} > 0$. Vamos a ver que 
    $\Bola{x}{\delta} \subset S_{\epsilon}(\omega)$. Para ello, tomemos
    $y \in \Bola{x}{\delta}$ y consideremos la siguiente $\omega_2 : [0,T] \to \mathbb{R}$:
    $$\omega_2(s) = \omega_1(s) + (y - x)$$
    \begin{itemize}
        \item En primer lugar, tenemos que $\omega_2(t) = \omega_1(t) + (y - x)
        = \omega_1(t) + (y - \omega_1(t)) = y$.
        \item Además, $\normi{\omega_1 - \omega_2} = |x - y| < \delta = \epsilon - \normi{\omega - \omega_1}$.
        \item Finalmente, $\normi{\omega - \omega_2} \leq \normi{\omega - \omega_1} + \normi{\omega_1 - \omega_2}
        < \epsilon$.
    \end{itemize}
    De todo esto sigue que $\omega_2 \in \Bola{\omega}{\epsilon}$ y que $y \in S_{\epsilon}(\omega)$,
    lo cual demuestra lo que deseábamos.

    \item
        \begin{itemize}
            \item Veamos primero que $\mathcal{B} \subseteq \mathcal{K}$.
            Al ser \CTesp \, un espacio métrico separable\footnote{Referencia 
            \href{http://mathoverflow.net/questions/46011/is-the-space-of-continuous-functions-from-a-compact-metric-space-into-a-polish-s}{acá}},
            por el ítem (b) sabemos que existen numerables bolas abiertas $\{B_n\}_{n \in \mathbb{N}}$ tales que
            $$\mathcal{B} = \sigma\left(\left\{B_n : n \in \mathbb{N}\right\}\right)$$
            Ahora bien, valiéndonos del ítem anterior, tenemos que cada $B_n \in \mathcal{K}$.
            De esto sigue que $\mathcal{B} \subseteq \mathcal{K}$.

            \item Ahora resta probar que $\mathcal{K} \subseteq \mathcal{B}$. Sea
            $B$ un abierto en $\mathbb{R}$. Vamos a ver que $U_t = \pi_t^{-1}(B)$,
            $t \in [0,T]$, es abierto en \CT, con lo que $U_t \in \mathcal{B}$. 
            Sea $x \in B$ y sea $\epsilon > 0$ tal que
            $\Bola{x}{\epsilon} \subset B$. Sea además $\omega_0 \in \CT$ tal que
            $\omega_0(t) = x \Rightarrow \omega_0 \in U_t$. Ahora consideremos
            una $\omega_1 \in \Bola{\omega_0}{\epsilon}$ arbitraria. Luego,
            $$|\omega_0(t) - \omega_1(t)| = |x - \omega_1(t)| \leq
              \normi{\omega_0 - \omega_1} < \epsilon$$
            En consecuencia, se tiene que $\omega_1(t) \in \Bola{x}{\epsilon} \subset B$,
            por lo que $\omega_1 \in U_t$.

            Para completar la prueba de $\mathcal{K} \subseteq \mathcal{B}$, podría
            razonarse por inducción transfinita en $\Bor{\mathbb{R}}$ argumentando
            lo siguiente:
                \begin{itemize}
                    \item Dado $B$ una unión numerable de conjuntos
                    $\{\tilde{B}_n\}_{n \in \mathbb{N}} \in \Bor{\mathbb{R}}$
                    tales que $\pi_t^{-1}(\tilde{B}_n) \in \mathcal{B}$ para todo $n$, se tiene que
                    $B \in \mathcal{B}$, y

                    \item Dado $B$ un complemento de cierto $\tilde{B} \in \Bor{\mathbb{R}}$ tal
                    que $\pi_t^{-1}(\tilde{B}) \in \mathcal{B}$, se tiene que
                    $B \in \mathcal{B}$.
                \end{itemize}
            
        \end{itemize}

\end{enumerate}

\end{proof}

%%%%

\begin{ej}
    (1.6 - Mörters y Peres). Sea $\{B(t) : t \geq 0\}$ un movimiento browniano
    standard. Probar que, casi seguramente,
    $$\lim_{t \to \infty}{\frac{B(t)}{t}} = 0$$ 
\end{ej}

\begin{proof}[Resoluci\'on]
Sea $X_i = B(t - i + 1) - B(t - i)$, $1 \leq i \leq \floor{t}$. Por ser $B$ un
movimiento browniano, se tiene que $X_1,\dots,X_{\floor{t}}$ son variables
aleatorias iid con $X_i \sim N(0, (t - i + 1) - (t-i)) = N(0,1)$. Luego,
valiéndonos de la Ley de los Grandes Números,

$$\frac{1}{\floor{t}} \, \sum_{i = 1}^{\floor{t}}{X_i}
    \stackrel{\textrm{c.s.}}{\longrightarrow} \textrm{E}[X_i] = 0$$

cuando $t \rightarrow \infty$. A partir de la definición de $X_i$, tenemos:

\begin{eqnarray*}
    \frac{1}{\floor{t}} \, \sum_{i = 1}^{\floor{t}}{X_i} 
    &=& \frac{1}{\floor{t}} \, \sum_{i = 1}^{\floor{t}}{B(t - i + 1) - B(t - i)} \\
    &=& \frac{1}{\floor{t}} \, \left(B(t) - B(r)\right) \\
    &=& \frac{B(t)}{\floor{t}} - \frac{B(r)}{\floor{t}} \\
    &\stackrel{\textrm{c.s.}}{\longrightarrow}& 0
\end{eqnarray*}

con $r = t - \floor{t}$. Pero $0 \leq r < 1$, con lo cual
$\frac{B(r)}{\floor{t}} \underset{t \to \infty}{\longrightarrow} 0$. 
De esto se desprende que necesariamente
$\frac{B(t)}{\floor{t}} \underset{t \to \infty}{\longrightarrow} 0$.
A su vez, esto implica que
$\left| \frac{B(t)}{\floor{t}} \right| =
\frac{\left|B(t)\right|}{\floor{t}}
\underset{t \to \infty}{\longrightarrow} 0$. Luego,
$$0 \leq \frac{\left|B(t)\right|}{t} \leq 
\frac{\left|B(t)\right|}{\floor{t}} \underset{t \to \infty}{\longrightarrow} 0$$

Se ve entonces que 
$\left| \frac{B(t)}{t} \right| =
\frac{|B(t)|}{t} \underset{t \to \infty}{\longrightarrow} 0$, de lo que se
puede concluir que $\frac{B(t)}{t} \underset{t \to \infty}{\longrightarrow} 0$,
que es lo que se prentedía demostrar\footnote
{Dada $f : \mathbb{R} \to \mathbb{R}$ tal que $|f(x)|
\underset{x \to \infty}{\longrightarrow} 0$,
$-|f(x)| \leq f(x) \leq |f(x)| \Rightarrow
f(x) \underset{x \to \infty}{\longrightarrow} 0$.}.


\end{proof}

\end{document}
