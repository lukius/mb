\documentclass[a4paper,11pt]{article}
\usepackage[utf8x]{inputenc}
\usepackage{fancyhdr}
\usepackage[spanish]{babel}
\usepackage{lastpage}
\usepackage{amstext}
\usepackage{amsmath}
\usepackage{amsfonts}
\usepackage{amsthm}
\usepackage{amssymb}
\usepackage{enumerate}
\usepackage{graphicx}
\usepackage{etoolbox}
\usepackage[a4paper, total={6.5in, 10.5in}]{geometry}
\usepackage[T1]{fontenc}
\usepackage[sc]{mathpazo}

\newcommand{\at}{@}

\title{Movimiento Browniano\\
      \small{Ejercicios entregables - Semana 1}}
\author{Lucio Santi\\
        \texttt{lsanti\at dc.uba.ar}}
\date{\today}

\pagestyle{fancyplain} 
\cfoot{\thepage/\pageref{LastPage}}
\lhead{}
\chead{}
\rhead{}

\newcommand{\norm}[1]{\left\lVert#1\right\rVert}
\newcommand{\CT}{\ensuremath{C([0,T], \mathbb{R})}}
\newcommand{\CTesp}{\ensuremath{(C([0,T], \mathbb{R}), \norm{\cdot}_{\infty})}}
\newcommand{\Bor}[1]{\ensuremath{\mathcal{B}(#1)}}
\newcommand{\floor}[1]{\ensuremath{\left\lfloor #1 \right\rfloor}}

\newtheorem*{ej}{Ejercicio}

\begin{document}
\maketitle

\begin{ej} 
    Este ejercicio es para caracterizar la $\sigma$-álgebra de Borel
    $\mathcal{B}$ en \CT.

    \begin{enumerate}[a)]
        \item Sea $(E,d)$ un espacio métrico separable y completo (polaco).
        Probar que todo abierto $U \subset E$ se puede escribir como unión
        numerable de bolas abiertas.
        
        \item Sea $(E,d)$ un espacio métrico polaco. Probar que existen 
        numerables bolas $B_1, \dots, B_n, \dots$ tal que la $\sigma$-álgebra
        de Borel \Bor{E} verifica
        $$\Bor{E} = \sigma\left(\{B_n : n \in \mathbb{N} \}\right)$$

        \item Para $\omega \in \CT$ definimos $\pi_t(\omega) = \omega(t)$.
        Probar que $\pi_t : \CT \rightarrow \mathbb{R}$ es continua.\\

        En \CTesp \, definimos la $\sigma$-álgebra de Kolmogorov,
        $$\mathcal{K} = \sigma\left(\left\{\pi_{t}^{-1}(B) : t \in [0,T],
        B \in \Bor{\mathbb{R}} \right\}\right)$$

        \item Probar que las bolas abiertas están en $\mathcal{K}$.

        \item Probar que $\mathcal{K} = \mathcal{B}$.
    \end{enumerate}

\end{ej}

\begin{proof}[Resoluci\'on]
TBD
\end{proof}

%%%%

\begin{ej}
    (1.6 - Mörters y Peres). Sea $\{B(t) : t \geq 0\}$ un movimiento browniano
    standard. Probar que, casi seguramente,
    $$\lim_{t \to \infty}{\frac{B(t)}{t}} = 0$$ 
\end{ej}

\begin{proof}[Resoluci\'on]
TBD
\end{proof}

\end{document}
