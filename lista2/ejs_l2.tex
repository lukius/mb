\documentclass[a4paper,11pt]{article}
\usepackage[utf8x]{inputenc}
\usepackage{fancyhdr}
\usepackage[spanish]{babel}
\usepackage{lastpage}
\usepackage{amstext}
\usepackage{amsmath}
\usepackage{amsfonts}
\usepackage{amsthm}
\usepackage{amssymb}
\usepackage{enumerate}
\usepackage{graphicx}
\usepackage{etoolbox}
\usepackage[implicit=false]{hyperref}
\usepackage[a4paper, total={6.5in, 9.5in}]{geometry}
\usepackage[T1]{fontenc}
\usepackage[sc]{mathpazo}

\newcommand{\at}{@}

\title{Movimiento Browniano\\
      \small{Ejercicios entregables - Semana 2}}
\author{Lucio Santi\\
        \texttt{lsanti\at dc.uba.ar}}
\date{\today}

\pagestyle{fancyplain} 
\renewcommand{\headrulewidth}{0pt}
\cfoot{\thepage/\pageref{LastPage}}
\lhead{}
\chead{}
\rhead{}

\newcommand{\abs}[1]{\ensuremath{\left\lvert #1 \right\rvert}}
\newcommand{\Sig}[1]{\ensuremath{\mathcal{#1}}}
\newcommand{\SigAlg}[1]{\ensuremath{\sigma\left(#1\right)}}
\newcommand{\Mart}[2]{\ensuremath{\left(#1_n, \Sig{#2}_n\right)}}
\newcommand{\Exp}[1]{\ensuremath{\textrm{E}\left[#1\right]}}
\newcommand{\ExpC}[2]{\ensuremath{\textrm{E}\left[#1 \, | \, #2\right]}}

\newtheorem*{ej}{Ejercicio}

\begin{document}
\maketitle

\begin{ej} 
    Sea \Mart{X}{F}~una martingala. Considerar $\Sig{U}_n = \SigAlg{X_1, \dots, X_n}$.
    Probar que \Mart{X}{U}~es una martingala.
\end{ej}

\begin{proof}[Resoluci\'on]
Veamos que $\Mart{X}{U}_{n \geq 1}$~satisface las tres propiedades que debe poseer
para ser una martingala. 

\begin{itemize}
    \item $X_n$ debe ser $\Sig{U}_n$-medible

    Por definición de $\Sig{U}_n = \SigAlg{X_1, \dots, X_n}$ se tiene que $\Sig{U}_n$
    es la menor $\sigma$-álgebra para la que $X_1,\dots,X_n$ son medibles. En
    particular, $X_n$ es $\Sig{U}_n$-medible.

    \item $\Exp{\abs{X_n}} < \infty$

    Esto sigue inmediatamente por ser \Mart{X}{F}~una martingala.

    \item $\ExpC{X_{n+1}}{\Sig{U}_n} = X_n$

    En primer lugar, observemos que $\Sig{U}_n \subseteq \Sig{F}_n$. Sabemos que
    $X_1,\dots,X_n$ son $\Sig{F}_n$-medibles por ser cada $X_i$, $1 \leq i \leq n$,
    $\Sig{F}_i$-medible y ser $(\Sig{F}_n)_{n \geq 1}$ una filtración (i.e.,
    $\Sig{F}_i \subseteq \Sig{F}_n$). Además, como ya se dijo, $\Sig{U}_n$ es la
    menor $\sigma$-álgebra para la que $X_1,\dots,X_n$ son medibles, de manera que
    $\Sig{U}_n \subseteq \Sig{F}_n$, que es lo que se pretendía observar. Luego,
    \begin{eqnarray*}
        \ExpC{X_{n+1}}{\Sig{U}_n} &\overset{\tiny{\textrm{(Torres)}}}{=}&
            \ExpC{ \ExpC{X_{n+1}}{\Sig{F}_n} }{\Sig{U}_n} \\
        &\overset{\tiny{(\Mart{X}{F}~\textrm{martingala})}}{=}&
            \ExpC{X_n}{\Sig{U}_n} \\
        &\overset{\tiny{(X_n \,\, \Sig{U}_n-\textrm{medible}})}{=}&
            X_n
    \end{eqnarray*}
\end{itemize}

\end{proof}

%%%%

\begin{ej}
    Sea $(X_n, \Sig{F}_n)_{n \geq 1}$ una martingala e $\{Y_n\}_{n \geq 1}$ un proceso tal
    que $\abs{Y_n} \leq C_n$ e $Y_n$ es $\Sig{F}_{n−1}$-medible. Sea $X_0 = 0$ y consideremos
    $$M_n = \displaystyle \sum_{k = 1}^{n}{Y_k \, (X_k − X_{k−1})}$$
    Probar que $(M_n, \Sig{F}_n)$ es una martingala. 
\end{ej}

\begin{proof}[Resoluci\'on]

\end{proof}

%%%%

\begin{ej}
Sea $B = (B(t), t \geq 0)$ un movimiento browniano unidimensional. Probar que es una martingala.
\end{ej}

\begin{proof}[Resoluci\'on]
Dado $t \geq 0$, sea $\Sig{F}_t = \SigAlg{B(s), s \leq t}$ ($(\Sig{F}_t)_{t \geq 0}$ es, pues,
la filtración natural de $B$). Veremos entonces que $B$ es una $\Sig{F}_t$-martingala probando
que satisface las tres propiedades enunciadas en la definición:

\begin{itemize}
    \item $B(t)$ es $\Sig{F}_t$-medible

    Esto es trivialmente cierto siendo $(\Sig{F}_t)_{t \geq 0}$ la filtración natural de $B$.

    \item $\Exp{\abs{B(t)}} < \infty$

    Sea $B(0) = x \in \mathbb{R}$. Siendo $(B(t) - B(0)) \sim N(0,t)$ tenemos que
    $$B(t) = \left(B(t) - B(0)\right) + x \sim N(x,t)$$
    Luego, $\abs{B(t)}$ tiene una distribución normal doblada, de manera que
    $\Exp{\abs{B(t)}} < \infty$ por ser $\Exp{B(t)} = x < \infty$.

    \item $\ExpC{B(t)}{\Sig{F}_s} = B(s)$, con $0 < s < t$

    Dado $0 < s < t$, 
    \begin{eqnarray*}
        \ExpC{B(t)}{\Sig{F}_s} &=& \ExpC{B(t) - B(s) + B(s)}{\Sig{F}_s} \\
        &\overset{\tiny{(\textrm{Linealidad de } E[\cdot|\cdot])}}{=}&
            \ExpC{B(t) - B(s)}{\Sig{F}_s} + \ExpC{B(s)}{\Sig{F}_s} \\
        &\overset{\tiny{(B(s) \,\, \Sig{F}_s-\textrm{medible}})}{=}&
            \ExpC{B(t) - B(s)}{\Sig{F}_s} + B(s) \\
        &\overset{\tiny(\star)}{=}& \Exp{B(t) - B(s)} + B(s) \\
        &\overset{\tiny{(B(t)-B(s) \, \sim \, N(0,~t-s))}}{=}&
            0 + B(s) \\
        &=& B(s)
    \end{eqnarray*}
    en donde $(\star)$ vale por independencia de la variable aleatoria $(B(t) - B(s))$
    y la $\sigma$-álgebra $\Sig{F}_s$, que a su vez se desprende de la independencia
    entre $B(t)-B(s)$ y $B(r)$, $0 \leq r \leq s$ (esto último como consecuencia de la
    propiedad de incrementos independientes de $B$).
\end{itemize}
\end{proof}

\end{document}
