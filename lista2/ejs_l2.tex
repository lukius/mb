\documentclass[a4paper,11pt]{article}
\usepackage[utf8x]{inputenc}
\usepackage{fancyhdr}
\usepackage[spanish]{babel}
\usepackage{lastpage}
\usepackage{amstext}
\usepackage{amsmath}
\usepackage{amsfonts}
\usepackage{amsthm}
\usepackage{amssymb}
\usepackage{enumerate}
\usepackage{graphicx}
\usepackage{etoolbox}
\usepackage[implicit=false]{hyperref}
\usepackage[a4paper, total={6.5in, 9.5in}]{geometry}
\usepackage[T1]{fontenc}
\usepackage[sc]{mathpazo}

\newcommand{\at}{@}

\title{Movimiento Browniano\\
      \small{Ejercicios entregables - Semana 2}}
\author{Lucio Santi\\
        \texttt{lsanti\at dc.uba.ar}}
\date{\today}

\pagestyle{fancyplain} 
\renewcommand{\headrulewidth}{0pt}
\cfoot{\thepage/\pageref{LastPage}}
\lhead{}
\chead{}
\rhead{}

\newcommand{\abs}[1]{\ensuremath{\left\lvert #1 \right\rvert}}
\newcommand{\Sig}[1]{\ensuremath{\mathcal{#1}}}
\newcommand{\SigAlg}[1]{\ensuremath{\sigma\left(#1\right)}}

\newtheorem*{ej}{Ejercicio}

\begin{document}
\maketitle

\begin{ej} 
    Sea $(X_n, \Sig{F}_n)$ una martingala. Considerar $\Sig{U}_n = \SigAlg{X_1, \dots, X_n}$.
    Probar que $(X_n, \Sig{U}_n)$ es una martingala.
\end{ej}

\begin{proof}[Resoluci\'on]
$ $

\end{proof}

%%%%

\begin{ej}
    Sea $(X_n, \Sig{F}_n)_{n \geq 1}$ una martingala e $\{Y_n\}_{n \geq 1}$ un proceso tal
    que $\abs{Y_n} \leq C_n$ e $Y_n$ es $\Sig{F}_{n−1}$-medible. Sea $X_0 = 0$ y consideremos
    $$M_n = \displaystyle \sum_{k = 1}^{n}{Y_k \, (X_k − X_{k−1})}$$
    Probar que $(M_n, \Sig{F}_n)$ es una martingala. 
\end{ej}

\begin{proof}[Resoluci\'on]

\end{proof}

\end{document}
